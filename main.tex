\documentclass[a4paper, twoside, openright]{book}

\usepackage[a4paper, top=2.5cm, bottom=2.5cm, left=2.5cm, right=2.5cm, centering]{geometry}

\usepackage[american]{babel}

\usepackage[bibstyle=authoryear, citestyle=apa, dashed=false, backend=biber, block=space, language=english]{biblatex}
\usepackage{csquotes}

\addbibresource{bibliography.bib}

\usepackage{graphicx}

\usepackage{subcaption}

\usepackage{hyperref}

\usepackage{titlesec}

\usepackage{fancyhdr}

\usepackage{emptypage}

\usepackage{amsmath}

\usepackage{float}

\usepackage{ragged2e}

\usepackage[toc]{appendix}

\linespread{1.5}

\setcounter{secnumdepth}{3}
\setcounter{tocdepth}{3}

\titleformat{\chapter}
{\normalfont\huge\bfseries}{\thechapter.}{0.5em}{\huge}

\titlespacing*{\chapter}{0pt}{-40pt}{25pt}

\titleformat{\paragraph}
{\normalfont\normalsize\bfseries}{\theparagraph}{1em}{\normalsize}
\titlespacing*{\paragraph}
{0pt}{3.25ex plus 1ex minus .2ex}{1.5ex plus .2ex}

\pagestyle{fancy}
\fancyhf{}
\lhead{\leftmark}
\fancyfoot[R]{\textbf{\thepage}}
\renewcommand{\headrulewidth}{0.4pt} 
\renewcommand{\footrulewidth}{0.4pt}
\setlength{\headheight}{15pt}

\fancypagestyle{plain}{
  \fancyfoot[R]{\textbf{\thepage}}
  \fancyhead{}
  \renewcommand{\headrulewidth}{0pt}
}

\hypersetup{
    colorlinks,
    linkcolor=black,
    filecolor=black,
    urlcolor=black,
    citecolor=black
}

%--------------------------------
\begin{document}

\begin{titlepage}

\begin{figure}[h]
    \begin{subfigure}[h]{0.2\textwidth}
        \includegraphics[width=\textwidth]{images/ub_logo.png}
    \end{subfigure}
    \hfill
    \begin{subfigure}[h]{0.2\textwidth}
        \includegraphics[width=\textwidth]{images/uab_logo.png}
    \end{subfigure}
    \hfill
    \begin{subfigure}[h]{0.12\textwidth}
        \includegraphics[width=\textwidth]{images/uni_girona.png}
    \end{subfigure}
    \hfill
    \begin{subfigure}[h]{0.2\textwidth}
        \includegraphics[width=\textwidth]{images/upf_logo.png}
    \end{subfigure}
    \hfill
    \begin{subfigure}[h]{0.2\textwidth}
        \includegraphics[width=\textwidth]{images/urv_logo.png}
    \end{subfigure}
\end{figure}


\begin{figure}[h]
    \centering
    \includegraphics[width=0.4\textwidth]{images/logo ccil.png}
\end{figure}

\vspace{4mm}
\begin{center}
  {\LARGE{MASTER IN COGNITIVE SCIENCE AND LANGUAGE}}
  \vspace{7mm}
  \\ {\large\textbf{MASTER THESIS}}
  \vspace{4mm}
  \\ {\Large{September 2024}}
\end{center}

\vspace{15mm}
\begin{center}
  {\LARGE\textbf{Specific contributions of cortical cortex on the integration of audio-visual walking inputs in patients with stroke compared to healthy subjects}}
  \vspace{15mm}
  \\ {\Large{by \textit{Emma Franchino}}}
\end{center}
\vspace{17mm}

\begin{center}
  {\Large{{Under the supervision of:}\\ \textit{Antoni Rodriguez-Fornells \\ Marta Matamala-Gómez}}}
\end{center}

\vspace{4mm}
\begin{figure}[h]
    \centering
    \includegraphics[width=0.15\textwidth]{images/Picture 1.jpg}
\end{figure}

\end{titlepage}

\cleardoublepage
\newpage 
\thispagestyle{empty}
\begin{flushleft}
  \vspace*{15em}
  \justify
  \textbf{Abstract}: The brain dynamics underlying sensorimotor synchronization constitute a complex combination of different neural processes. Here we wanted to investigate the contribution of cortical cortex on the integration of three different sensory stimuli: visual, auditory and audiovisual. The study was conducted by means of electroencephalography (EEG) in a frequency-tagging approach on two distinct groups of individuals: healthy control subjects (\textit{N = 21}) and patients with a stroke (\textit{N = 21}). Both population groups showed neural responses of the sensorimotor area in all the stimuli presented in a rhythmic condition, though not in the ones displayed in a random one. Such outcome is sustained by behavioral questions on the positive feeling towards the distinct stimuli, which appeared with significantly higher values for the inputs repeated in the rhythmic condition. Moreover, in the stroke population different response patterns could be observed in relation to the intensity of the neural activation, which resulted weaker than in the control group, and in the left-lateralized response, which was not found in the healthy subjects. These results demonstrate that the neural entrainment and activation of the sensorimotor area can be elicited only by sensory stimuli repeated in a rhythmic sequence and that auditory inputs appear better to induce neural tracking. Furthermore, it could be concluded that stroke patients preserved the ability of sensory integration and neural entrainment, even if it results reduced compared to healthy individuals. 
  \vspace*{3em}
  \justify
  \textbf{Keywords}: EEG, Frequency-tagging, Stroke, Sensorimotor integration
\end{flushleft}

\tableofcontents

\chapter*{Acknowledgements}
I am indebted to my advisors Marta Matamala-Gomez and Antoni Rodriguez-Fornells, for the guidance and precious suggestions that they gave me during the whole process of collecting and analyzing the data, as well as the writing of the thesis. \\
I also would like to thank David Cucurell Vega from the Cognition and Brain Plasticity Unit for his precious support and patient during the EEG analysis. \\
The thanksgiving is also extended to the whole laboratory of Bellvitge for the warm welcoming and the important insights, which helped with the research. \\
Then, I want to thank my friends, from the ones of a lifetime to the ones who I just met this year, who make life funnier and never make me feel lonely. \\
I give my most profound thank you to my family, who supported me in every step of my academic carrier and always make me feel loved. \\ A huge gratitude and love goes to my Umbi who has helped me during the process of this thesis in both technical aspects, and especially in the emotional ones. \\
Finally, this work is dedicated to nonno Piero, who I know would be proud.

\chapter{Introduction}
%%% general introduction on stroke
%% stroke incidence
An average of 5.5 million people worldwide die every year because of a stroke, ranking it as the second leading cause of death and one of the main cause of cardiovascular diseases \parencite{Donkor_2018}. Research in the United States has stated that more than 795,000 people have a stroke every year, that is almost one person every forty seconds \parencite{Tsao_2023}. In the past three decades, an increase in the incidence of stroke was globally observed, especially in young (18-44 years old) and mid-life (45-64 years old) populations, yet not in older adults (over 65); on the other hand, the incidence of death and disability has decreased as a result of the advance of research and medicine \parencite{Yahya_2020}. \\
Even though a decrease in disability has been observed, stroke remains the leading cause of serious long-term disability: more than 50\% of people who survived a stroke become chronically disabled, both in neurological and physical aspects. One of the most common consequences is long-term motor impairment due to lesions in cortical or subcortical motor areas \parencite{Karthikeyan_2019}. 

%% stroke: ischemic and hemorrhagic // cortical subcortical
Stroke is divided in two main subtypes, drawn by the cause of it. The most common one results to be the ischemic stroke, it's produced by a blockage of blood vessels and comprehends TIAs (transient ischemic attacks) or mini strokes, which only last for a few minutes up to 24 hours. The second type is the hemorrhagic stroke which is caused by bleeding in or around the brain, and it appears to cover just the 15\% of the cases \parencite{Abdu_2021}. \\
On the other hand, if we look at the brain regions lesioned by stroke, it's possible to differentiate it between cortical and subcortical injuries, which can alter different cognitive (e.g. language, space orientation) and physical abilities (e.g. motor impairment, weakness), depending on the specific cerebral area and lobe affected. 

%% Gait impairment
One of the major consequences of experiencing a stroke is impaired gait: a total of 80\% of patients have an impaired walking ability, with 50\% of patients who are completely unable to walk, 12\% who can walk with assistance, and 37\% can walk independently \parencite{Balaban_2014}. In those cases, gait can result as hemiplegic, slower in speed and cadence and as well as shorter stride length \parencite{Gomez_2020}. The balance of the locomotor control is shift due to the lesion in the central nervous system (CNS) affecting automaticity, proprioception, balance, anxiety state and multisensory feedback, which will impair walking abilities \parencite{Clark_2015}. 

Due to the wide spreading of gait impairment in patients who suffered from a stroke, the recovery of walking ability is of the main aims in stroke rehabilitation. Recently rehabilitation techniques have been moving towards top-down approaches, which base the rehabilitation on the state of the brain after stroke. These techniques use a motor learning approach mainly through the integration of sensory feedback, helping the proprioception of the body thanks to sensory-motor adaption \parencite{Belda-Lois_2011}. \\
The human brain presents the function of multisensory integration: it integrates the information of different sensory modalities into a coherent representation \parencite{Stein_2008}. Multisensory integration can improve motor responses, through the ability of the brain to extract sensory information regarding body position and anticipate future positions. Additionally, it has been observed that many multisensory processes appear to be largely preserved in stroke patients, keeping integrate the ability to synthesize multisensory information representing motor activity \parencite{Bolognini_2013}.

From on side, research has investigated the usage of different sensory inputs for post-stroke gait rehabilitation: for what concerns visual stimuli, it has been shown that Mirror Neuron System (MNS) play a crucial role in action observation and motor learning, through the stimulation of an internal model, able to reinforce sensorimotor representation \parencite{Rizzolati_2004}. Furthermore, visual feedbacks used in motor imagery-based training can also improve motor recovery in post-stroke patients, especially the ones affected by severe hemiparesis \parencite{Mihara_2013}. \\
On the other hand, auditory feedback has been a fundamental sensory feedback for gait rehabilitation, being used as a real-time cue to correct movement and redirect walking ability. Sounds can activate excitability through auditory-motor circuits at the reticulo-spinal level; if they are repeated rhythmically, they can entrain the timing of muscle activation, enhancing movement during rhythmic actions \parencite{Thaut_1999}. \\
Lastly, the employment of audiovisual stimuli has been explored: MNS appears to respond to the visual representation of action, as well as the auditory one \parencite{Rizzolati_2004}; the integration of both sensory modalities can result more beneficial considering the positive effects of multisensory integration, both in healthy and clinical populations \parencite{Bolognini_2015}. \\
Both auditory and visual stimuli have been employed in different neurophysiological rehabilitation techniques, such as in Proprioceptive Neuromuscular Facilitation (PNF) \parencite{Moros_2000} or Motor Imagery (MI) \parencite{Mason_2007}. 

Some evidence has found auditory feedback as fundamental for walking rehabilitation: it reacts faster than the visual system and can continuously be delivered without constraining the movements. From such discoveries, movement sonification has been engaged for these goals; here the physical properties of the human movement are being translated to audio parameter (e.g. the speed or position of the body are translated into volume or timbre of audio sounds) \parencite{Effenberg_2016}, enhancing motor learning based on motor perception and its representation \parencite{Bevilacqua_2016}. \\
Still, little research on movement sonification for sensorimotor learning and movement recovery has been done, even more the study of the neurological processes underlying such phenomena in clinical population.

Sensory feedback employed for movement rehabilitation induce biological motion perception. Studying the neurological processes that underline motion perception, results as a complex goal, since multiple social cues perception and process are involved. In order to isolate only the interested brain activity related to motor perception a new paradigm studying neural entrainment through electroencephalogram with a frequency-tagging approach, with a neutral visual stimulus of a point-light figure walking, has been proposed by \cite{Cracco_2022}. 

%% neural entrainment: definition
Neural entrainment or sensorimotor synchronization is defined as unidirectional synchronization of neural oscillations to an external rhythmic stimulus \parencite{Lakatos_2019, Haegens_2018}; an action is rhythmically coordinated with a predictable external stimulus \parencite{Pressing_1999}. We use rhythmic synchronization between the auditory and motor systems in many of our everyday behaviors, one of the main being speech. Our brain leans towards a natural synchronization of its movement with endogenous rhythmic signals: auditory rhythms rapidly entrain motor responses into stable steady synchronization states \parencite{Thaut_2003}. Such alignment of motor and sensory rhythm at the neural level can be detected in music, dance, sports, rhythmic tasks and verbal communication \parencite{Rosso_2023}. Sensorimotor entrainment can be driven by a status of phase- and frequency-locking of neural oscillations, which will reflect the changes in the external sensory through neuronal excitability \parencite{Lakatos_2005}. 

%% frequency tagging approach (what is it and why it looks like the best methodology)
A way that has been found successful to study neural entrainment is the frequency-tagging approach, where a brain response is elicited through a repeated cycle movement; if the aim is movement perception related to walking the brain response can be elicited at the end of every footstep \parencite{Cracco_2022}. The cognitive and sensory processes present in neural entrainment (brain synchronization with the sensory input presented) are represented by neural oscillations that can be observed through electroencephalography (EEG): EEG signals result in time-locked to re-occurring sensory stimuli \parencite{Thut_2012}. \\
State-of-the-art approaches tag frequencies of a rhythmic stimulation in the power spectrum of EEG in order to quantify neural entrainment \parencite{Nozaradan_2011}: the "frequency tagging" methodology converts brain signals into their frequency components. 

%% rhythmic audio and visual stimuli to activate sensorimotor cortex
The usage of frequency tagging with electroencephalography results in the most appropriate technique to study the role of different brain regions integrating multisensory stimuli through neural entrainment, being able to isolate the brain processes related to movement perception, distinguishing them from other cognitive processes that may be related to social cues using neutral sensory inputs, such as a point-light figure with no gender, weight nor emotional trait \parencite{Cracco_2022}. \\
Brain synchronization remains a complex phenomenon, previous research showed that listening to repeated rhythmic beats activates cortical oscillation evoked from auditory stimuli and anticipations of it \parencite{Snyder_2005}. The different brain regions that are being enhanced during auditory-motor entrainment include: the cerebellum \parencite{Grahn_2011}, inferior colliculus \parencite{Tierney_2013}, basal ganglia \parencite{Thaut_2009} and cortical hubs, located at the top cortical networks, that enable sensorimotor integration. \\
%% aim of the study
In conclusion, research proved that the auditory and visual feedback can elicit motor responses by neural entrainment in the motor cortex, generating synchronized brain oscillations to the rhythm of the stimuli; furthermore different studies demonstrated that such feedbacks play a key role in stroke gait rehabilitation \parencite{Chen_2018,Bolognini_2016}, facilitating motor automaticity, redirection and action. 

Even if more research in gait rehabilitation has been done in patients with stroke, few investigations have examined the neural activation related to the integration of multisensory stimuli (audio and visual) related to walking movement. Hence, the aim of the present study is to investigate the brain dynamics supporting the sensorimotor synchronization when coupling auditory (footstep sounds) and visual (walking point-light-figure) inputs (see Figure \ref{fig: visual stimuli}) related to walking movement in a group of patients with stroke compared to a group of healthy subjects. \\
The study will be conducted using an electroencephalogram frequency tagging approach, aiming to elicit neural synchronization between brain oscillatory activity and the presented sensory inputs (visual: walking point-light-figure; and auditory: footstep sounds) related to walking ability presented at 2 Hz. The frequency rate presentation of the stimuli was selected based on previous study conducted in our laboratory by Matamala-Gomez et al. in which the authors observed that at 2 Hz when presenting auditory and visual inputs at rhythmic sequences there was a higher synchronization with the sensorimotor cortex in healthy subjects (Matamala-Gomez et al. in preparation). In detail, in the previous study the authors compared different frequency rates on the presentation of the sensory inputs (1, 2, and 3.6 Hz) in rhythmic and random sequences, and observed a higher synchronization with the temporal and sensorimotor cortex when presenting auditory (footstep sounds) and visual (walking point-light-figure) at 2 Hz in rhythmic sequences. Moreover, the authors also observed a higher synchronization with the Occipital cortex when presenting the visual input (walking point-light-figure) at 2 Hz in rhythmic sequences. Based on the previous results, in the present study we expect to find a higher brain synchronization when presenting auditory inputs (footstep sounds) at 2 Hz in rhythmic sequences, with the temporal and sensorimotor cortex. Moreover, we also expect to find a higher synchronization when presenting the visual inputs (walking-point-light figure) at 2 Hz in rhythmic sequences with the occipital cortex and a weaker elicitation of the sensorimotor cortex. For all the sensory inputs we predict to see a significant higher neural synchronization in the stimuli a rhythmic sequence compared to the random sequences. \\
Finally, regarding the patients with stroke, we expect to observe some differences in brain synchronization compared to the healthy control group. Since cortical and subcortical areas were depicted as facilitators for motor alignment thanks to the ability of beat-based time-keeping \parencite{Cannon_2021}, we expect differences of brain activation in stroke patients compared to healthy subjects, but still, visible neural entrainment elicited by repeated rhythmic multisensory stimuli.

%%% --- poster Marta
% Frequency tagging measures a periodic change in voltage amplitude in the electrical activity recorded on the human scalp by using EEG when presenting stimuli repeated at a fixed rate1. Regarding auditory stimulation, some studies show a tonic synchronization response when using periodic auditory stimulation set at 2Hz2. Further, others found brain responses associated with biological motion perception by showing a point-light walker video moving at a pace of 2.4Hz3.
% Movement sonification implies the mapping of movement signals into sound, that can be used for motor rehabilitation4. To date, none have investigated frequency tagging responses and the neural correlates related to movement sonification techniques for motor rehabilitation.
% Aim of the study
% In this study we aimed to evaluate the impact of sonification on walking ability as indexed by neural markers and brain dynamics on the sensorimotor, the auditory, and the visual cortex using a frequency tagging approach in healthy subjects.

\chapter{Methods}
\section{Participants}
In the present study, we had a total population of 42 subjects, equally divided into two groups: 21 individuals were part of the healthy control group, whilst the remaining 21 were patients who suffered from a cortical or subcortical stroke. The recruitment for the control group was made through several methods: in person, talking to alumni of the \href{https://www.experiencia.ub.edu/ca/}{Universitat de l'Experiència of Barcelona}, through personal contacts, and using the platform \href{https://www.sona-systems.com/}{Sonasystem}. \\
On the other hand, to recruit participants who experienced a stroke we used a database previously employed in another study made in the \href{https://brainvitge.org/}{Cognition and Brain Plasticity Unit} for movement rehabilitation with music therapy. Furthermore, we were also able to present our study at \href{https://www.fundacioictus.com/}{Fundaciò Ictus}, where we could select more participants. \\ 
The exclusion criteria in both groups were: no visual or auditory impairments and no pregnancy.

Participants of the two groups were matched by age and gender, where we can find a mean of 55,4 years old in the control group and of 56,4 for the stroke patients, with a non-significant difference (\textit{t} test: -0.299, \textit{p} value: .766). Likewise, the difference is gender didn't result significantly different (\textit{t} test: 1.56, \textit{p} value: .127), even though a majority of men in the stroke group and a preponderance of women in the control group could be observed. \\ To both groups a questionnaire measuring the level pf physical activity was administrated: an average of 7026,78 kilojoules a day was found for the control group, whilst for the stroke group it resulted of 5669,19 kJ/d, which results lower than the control group, but still without a significant difference (\textit{t} test: 1.06, \textit{p} value: .146). \\
For what concerns the stroke group, the clinical history of the patients was collected, founding the majority of patients with the injured left hemisphere, which affected their contralateral right extremity (\textit{N=}12 suffered the injury in the left hemisphere; \textit{N=}7 in the right hemisphere; \textit{N=}1 in both hemispheres and \textit{N=}1 is missing).\\ 
Participants signed an informed consent before the experiment, and they were reimbursed for their time. 
% This study was conducted according to the ethical rules presented in the General Ethical Protocol of the Faculty of Psychology and Educational Sciences.

\section{Materials}
The experiment was conducted with electroencephalography, recorded using a standard set-up with 64 Ag-AgCl electrodes placed on the scalp according to the International 10/10 system (\href{https://www.easycap.de/}{EasyCap}) to capture the neural activity in the whole brain. \\
The study presented auditory and visual stimuli related to the walking ability, all set at the frequency of 2 Hz, using Matlab for the visual stimuli and \href{https://www.audacityteam.org/}{Audacity software} for the auditory ones; resulting the best to activate neural entrainment (Matamala et al. in preparation).

For what concerns the visual stimuli, a point-light figure walking, representing the human body through white dots set in the primary joints of the body (Figure: \ref{fig: visual stimuli}), was used: it was created with \href{https://www.biomotionlab.ca/html5-bml-walker/}{BioMotionLab}, using a neutral subject, without any specific gender nor emotional display, with an average weight. After being created with the mentioned software, its frequency was adjusted to the desired one. The choice of a neutral point-light figure as the visual stimulus was made to focus only on brain activity elicited by movement processing frequency. Earlier studies have in fact shown that the presence of different important biological and social features that can be inferred from figures may enhance and grab other processes not primarily related to movement perception \parencite{Cracco_2022}. \\
For the audio stimuli, the sound of footsteps at 2 Hz with a neutral connotation was engaged (Figure: \ref{fig: audio stimuli}). The participants listened to the sound through headphones and at contemporary they were asked to watch a blue fixation cross presented in the middle of the black screen in front of them in order to keep their eyes opened. \\
Finally, a third stimulus consisting of the union of the point-light figure walking and the sounds of the footsteps, was employed and denominated as audiovisual. 
\begin{figure}[h]
    \centering
    \begin{subfigure}[h]{0.4\textwidth}
        \centering
        \includegraphics[width=0.35\textwidth]{appendix/point_light_figure.png}
        \caption{Point-light figure used for visual condition}
        \label{fig: visual stimuli}
    \end{subfigure}
    \hspace{4em}
    \begin{subfigure}[h]{0.4\textwidth}
        \centering
        \includegraphics[width=0.85\textwidth]{appendix/audio_images.png}
        \caption{Rhythmic and random audio stimuli}
        \label{fig: audio stimuli}
    \end{subfigure}
    \label{fig: stimuli}   
\end{figure} \\
Moreover, six behavioral questions were presented (they can be found in Figure \ref{fig: Behavioral questions}): they referred to how the participant felt while seeing the point-light figure walking or listening to the footsteps sound. The questions had to be answered on a Likert scale with values between 1 (strongly disagree) to 5 (strongly agree). \\
The experiment was structured with the software \href{https://pstnet.com/products/e-prime/}{E-prime}, which allowed us to build an experiment with different stimuli presentations (e.g. adding images, questions and text).

To measure the level of physical activity of each participant, which would be later correlated to the EEG results, we used the \textit{Active-Q} questionnaire, which was created precisely to assess the total physical activity and inactivity in adults, through a series of multiple choice questions (referred to a one-year time) on daily activity, mean of transportation used, leisure and sports activities; and the time a day spent for each of them \parencite{Bonn_2012}. \\
The questionnaire was translated in Spanish from the original English version and transformed from a PDF file into an interactive digital version using the platform \href{https://www.qualtrics.com/uk/?rid=ip&prevsite=en&newsite=uk&geo=ES&geomatch=uk}{Qualtrics}. 

\section{Experimental design}
The current study is a mix models study design presenting both auditory (footstep sound) and visual (walking point-light figure) inputs at the frequency rate of 2 Hz in a rhythmic or random sequence, in three different experimental blocks: i) Auditory task, ii) Visual task and iii) Audiovisual task. The blocks were presented in a randomized order to all the participants. \\
At the beginning of our experimental session we included a text with the instructions and right afterwards a white fixation cross preceding each stimulus, which lasted five seconds and gave time to the participant to prepare themselves. 

In each experimental block the stimuli were repeated eight times: they were counterbalanced between participants being presented in a random order by their synchronicity or asynchronicity, and the number of variations used as a focus tool. These refer to the changes of the pitch (to a more acute or grave sound) in the sound stimuli and the switch of color in the central point of the point-light figure (from red to white) in the visual and audiovisual stimuli. \\
After each stimulus the behavioral questions related to the correspondent sensory inputs were displayed and right after them a figure of an eye-blink was shown, which lasted fifteen seconds and allowed the participant to blink if needed.

Individually each stimulus lasted one minute, therefore including the eye-blink image and the behavioral questions, each experimental block lasted approximately fifteen minutes (depending on how long the participants would take to answer the behavioral questions). A break of roughly five minutes was taken between all blocks to fix the conductivity of the electrodes with a gel \footnote{The gel employed in the EEG procedure is a mix of salts and water that allows the conductivity between the scalp and the electrodes.} and to let the participants rest. Altogether, the experiment and the set-up of the EEG cap on the participants had a duration of almost two hours.

For what concerns the Active-Q questionnaire, it was administrated via e-mail a few days before the experiment, together with some additional information about the study.

\section{Procedure}
Before the experiment began the participants had to sign the informative consent, and they would ask questions if needed. Afterwards a careful preparation was made, where we placed the EEG cap on their head\footnote{The EEG cap was previously prepared with all the 64 electrodes in their relative site.}, the external on the side of the right eye and under it, as well as on the mastoids. Then a gel was inserted in each electrode (both external and placed on the cap) in order to diminish the impedance and allow a good signal connectivity. \\
Additionally, if the participants had any trouble completing the Active-Q questionnaire at home, they were helped to do so before the start of the experimental session. 

The participants sat on a chair in front of the computer screen where the stimuli were displayed, and wore headphones to listen to the auditory stimuli. They were asked to avoid moving and blinking when the sensory stimuli were presented to avoid noise signal in the EEG recording. \\
After reading the instruction on the screen, the participants had to pay attention to the stimulus that was being presented and count the number of changes in the color of the point-light figure's central dot or in the tone of the footstep sounds. At the end of each stimulus, they were asked to tell the experimenter how many changes they were able to observe or hear through a speaker\footnote{The speaker was used to communicate with the participants, which were in a different room from the experimenters. Furthermore, they could be seen through a webcam, and make signs if anything was needed.}. 

Afterwards, the behavioral questions were presented and had to be answered by the participants using numbers from 1 to 5 on the keyboard placed in front of them. When a question was answered the following one was automatically shown. When all the six questions were completed the eye-blinking image was presented, this was inserted for the participants to yawn, blink or stretch their face muscles if needed, furthermore they were expressively told that they could do so also during the answering of the questions, since the EEG recording during those time was of no interest for our analysis.  

After all the eight stimuli with their relative questions were shown, a block was completed and a small break took place, where the participants were offered something to drink or eat and the electrodes' impedance was checked. \\
When all the blocks were completed, the participants could wash their head from the gel in a sink.

\section{Analysis}
\subsection{EEG data analysis}
The EEG data was registered through the software provided by \href{https://brainvision.com/applications/brain-vision-software/}{BrainVision} at the sampling rate of 1000 Hz; for all the subjects we made three different registrations one per condition determined by the stimuli (visual, audio, audiovisual). \\
Before the actual frequency-tagging analysis, the data was preprocessed using the \href{https://eeglab.org/}{EEGLAB toolbox} run with Matlab: all the chunks of register that weren't necessary (i.e. the initial part with the instructions, the one related to the eye-blink figure and the ones corresponding to the behavioral questions) were trimmed. Later we re-referenced the data to channels 31 and 32, which correspond to the electrodes TP9 and TP10, positioned on the mastoids. Finally, we ran the Independent Component Analysis (ICA) to remove artifacts and saved the preprocessed data into a new file. 

After the data for every block for all the participants was preprocessed, we started the analysis with the help of some scripts made for the previous study adjusted according to our necessities. \\
Firstly, we extracted the epochs from the processed data: epochs of 20 seconds were obtained from the 60 seconds of each stimulus; the method of surface Laplacian and current-source density (CSD) transform was applied to the data, dividing it between the one related to the rhythmic stimuli and the random ones. \\
From the files just generated, we computed spectral power activity with \href{https://www.fieldtriptoolbox.org/}{FieldTrip toolbox} of Matlab, using \textit{mtmfft} method, which allowed performing later analysis in time and frequency domains to study oscillatory brain activity and its modulation over time. The Fast Fourier transform (FFT) was applied to each stimulus (visual, audio and audiovisual) in the two different conditions of synchronicity and asynchronicity (also referred to as rhythmic and random); therefore six variables were obtained. 

Using these variables we created the desired graphs, showing how the activity power would change through conditions, stimuli and channels in order to see if we could find peaks of activity correspondent to the frequency used for the stimuli. We plotted the averaged Signal-to-Noise Ratio (SNR) and the activity power spectrum to see how the neural activity would change through frequencies; if the desired peak amplitude activity at 2 Hz could be found. \\
For the activity power spectrum analysis we decided to take into considerations the different regions of interest (ROIs) of our experiment in order to see how the amplitude activity would manifest there. The ROIs concerned the Occipital, Temporal and Sensorimotor areas, respectively enhanced by visual and audio sensory inputs and motor synchronization.

Finally, our analyzed data was used to create the desired topographies, both two- and three-dimensional, showing the cerebral activity for each sensory stimuli, divided into rhythmic and random conditions, adding the ones to represent the difference between these two. In the two-dimensional topographies we decided to distinguish the electrodes corresponding to the different regions of interest coloring them in white. \\
The analysis just described, including all the graphs and images was made separately for the control group and stroke patients populations. 

\subsection{Behavioral data analysis}
For what concerns the analysis of the behavioral questions, we searched for all the answers given by the participants, which were collected into a \textit{.txt} file by E-prime, through a Python script. This would insert each numeric answers into an Excel file, so that later it could be used for statistic operations. \\
For each question of every stimulus, specifically differenced into rhythmic and random conditions and population group, we calculated its average, standard deviation, and significance through a Wilcoxon Signed-Rank test if the data resulted non-normally distributed, and a paired T-test if it resulted otherwise. In order to understand the distribution of every set of data we earlier performed the Shapiro–Wilk test. \\
We inserted all of our statistic results for each group and sensory input into different Excel tables in order to build with them separate bar plots (with their corresponding error bars) to show graphically the comparison between the rhythmic and random condition. \\
Finally, we decided to achieve an additional comparison between the rhythmic and random condition in the two distinct population groups, building a single bar plot with the mean results of all the questions in the diverse stimuli in the two groups. 

%% ---- questionnaire analysis part 
Regarding the Active-Q questionnaire we conducted the analysis following the instruction reported on the article where it is explained, and its efficiency tested \parencite{Bonn_2012}. Initially, we exported in an Excel file the results collected by Qualtrics with their numeric values (e.g. if the participant had selected the first option, the result would have been 1), this was imported in a Python file where the numeric values related to each activity were transformed into the MET values reported in the table \ref{fig: met_values}, that were encountered in the previously cited article. On the other hand, the frequency of each activity was measured multiplying the hours/minutes a day for the number of days per week; if a range of time was presented we previously calculated the average of it, and convert it into hours a day (e.g. the range 15-29 minutes/day was converted to 0.36 h/d).  \\
After creating an Excel table with the new values accorded to MET values and the duration of hours/week, we proceeded to calculate the final Active-Q value, which returns a number of kilojoules per day, using the following formula: 
\[
\text{EE}_{\text{activity}} \, (\text{kJ/day}) = \text{MET}_{\text{activity}} \cdot \text{Weight} \, (\text{kg}) \cdot \text{Duration}_{\text{activity}} \, (\text{h/d}) \cdot 4.184
\]
The MET value and duration of the activity were multiplied by the weight of each participant (which was asked individually to the participant before the experiment), and the number 4.184, which is a conversion factor used to transform values from kcal to kJ. The obtained results represented the daily physical activity of each participant. 

The physical activity as well as the behavioral questions were aimed to be correlated with the brain activity power registered through EEG to see if the level of physical activity of the individuals could influence the neural entrainment observed; or if the feeling towards the inputs of the participants could be correlated with the detected brain activity. \\ 
In order to obtain a numerical value for each participant and condition representing the neural activity experienced perceiving the different stimuli, we extracted the mean power activity of the different ROIs for each sensory inputs and participant in the two population groups. 

We then proceeded to insert the obtained values in an Excel file together with the Active-Q results and the behavioral questions ones to later calculate their Pearson's correlation coefficient and its relative \textit{p} value and build different scatter plots. \\
For the behavioral questions we have to precise that we decided to take into account for this correlation just the second question (q2) for each stimulus, because considered as the most significant, since it is strictly related to the rhythm perception.

\chapter{Results}
\section{EEG results}
The outcome of our EEG analysis are descriptive EEG results, which will be deepened in later studies, through advanced statistical analysis.\\ 
Firstly, we confirmed the neural rhythm synchronization of the different ROIs to the frequency of 2 Hz for every sensory inputs, calculating the averaged signal-to-noise ratio (SNR) and the mean power spectrum and its variance through the different frequencies. We could actually observe an activity peak related to the 2 Hz frequency in the sensory inputs presented with a rhythmic sequence, and spread activity peaks in the random condition for both population groups, even if a lower power amplitude at 2 Hz could be detected in the stroke group. 

These results are better depicted by topographies representing the activity power spectrum only at the frequency of 2 Hz 
related to the three different stimuli, for both the rhythmic and random conditions and the difference between these two. Different graphs were built for each population group. \\
In Figure \ref{fig: 3D topographies control group} we can analyze the neural activity of the control group: a strong activation of the temporal and sensorimotor areas were observed both in the audio and audiovisual rhythmic conditions, with a broader spread of the activity in the surrounding areas in the audio condition and the supplementary activation of the occipital lobe in the during the audiovisual stimuli. For what concerns the brain activation during the rhythmic visual stimuli, it can be seen clearly only in the occipital area. \\
Moving to the topographies related to the random condition, the pattern of the brain activation appears as much weaker than the rhythmic condition, with a feeble arousal in some different brain regions. \\
We can conclude that the frequency tagging effect is greater in the rhythmic conditions, with an activation of the ROIs that can be detected through the difference between the two conditions. 

For what interests the stroke population, the encountered results (Figure \ref{fig: 3D topographies stroke group}) appear as slightly different from the control group. During the audio stimuli, the temporal and sensorimotor areas look activated, even if not as much as in the control group. Furthermore, strong lateralized arousal on the left side of the brain can be denoted in both the topographies related to the audio and audiovisual stimuli. On the other hand, the visual stimuli elicited a weaker activity in the stroke population compared to the control group, as well as a softer left sided activation than the audio and audiovisual ones. \\
As in the control group, the random condition present a general weak brain activation, with the main difference of a noticeable arousal for what concerns the left side lateralization. \\
Finally, the difference between the two conditions confirms the activation of the diverse ROIs related to the corresponding sensory inputs, resulting more feeble than the ones found in the control group. 
\begin{figure}[htbp]
    % \centering
    \begin{subfigure}[htbp]{1.1\textwidth}
        \includegraphics[width=\textwidth]{healthy_images/3d_topo.png}
        \caption{3D topographies related to the activity in the control group}
        \label{fig: 3D topographies control group} 
    \end{subfigure}  
    \vfill
    \begin{subfigure}[htbp]{1.1\textwidth}
        \includegraphics[width=\textwidth]{stroke_images/3d_topographies.png}
        \caption{3D topographies related to the activity in stroke population}
        \label{fig: 3D topographies stroke group}   
    \end{subfigure}
\end{figure} 

\clearpage
\section{Behavioral results}
For the analysis of the behavioral results, we firstly made various bar plots divided for population groups and sensory inputs. Firstly, in Figures \ref{fig: bar_visual_control}, \ref{fig: bar_audio_control} and \ref{fig: bar_audiovisual_control}, we can find the results related to the control group: here we can see that all the questions related to every sensory stimulus differ significantly between the random and rhythmic conditions, with higher values for the rhythmic condition. These results are confirmed by the report \textit{p} value in the statistical table in Figure \ref{fig: significance_control_pop}. \\
From the graphical representation and the statistical results we can also detect that the question with the higher values in the rhythmic condition results to be the third one in all the stimuli, which refers to the perception of the movement fluidity by the subject. While this question appears as the one expressing the biggest difference between the two conditions, the last three questions in every sensory inputs (q4, q5 and q6) emerge to be the ones with lower differences between random and rhythmic, still it being significant.
\begin{figure}[htbp]
    \begin{subfigure}[htbp]{0.5\textwidth}
        \centering
        \includegraphics[width=\textwidth]{bar_plots/plotbar_visual_h.png}
        \caption{Results of behavioral question in the visual condition in the control group}
        \label{fig: bar_visual_control} 
    \end{subfigure} 
    \begin{subfigure}[htbp]{0.5\textwidth}
        \centering
        \includegraphics[width=\textwidth]{bar_plots/plotbar_audio_h.png}
        \caption{Results of behavioral question in the audio condition in the control group}
        \label{fig: bar_audio_control} 
    \end{subfigure} 
    \begin{subfigure}[htbp]{0.5\textwidth}
        \centering
        \includegraphics[width=\textwidth]{bar_plots/plotbar_audiovisual_h.png}
        \caption{Results of behavioral question in the audiovisual condition in the control group}
        \label{fig: bar_audiovisual_control} 
    \end{subfigure} 
\end{figure}

Moving on to the outcome of the questions answered by the stroke group, which can be found in Figures: \ref{fig: bar_visual_stroke}, \ref{fig: bar_audio_stroke} and \ref{fig: bar_audiovisual_stroke}, similar results to the control group can be observed, with some minor differences. Firstly, we can affirm that equally to the previous results, a significant difference between all the questions related to the random and rhythmic for each sensory input, can be found (Figure \ref{fig: significance_stroke_pop}). \\ However, in the stroke population we can see that the difference appears to be generally lower compared to the one observed in the control group, except for question three in all the sensory inputs, question four in the audio stimulus and five in the audiovisual one. In general, we can conclude that the Likert values assigned to the questions referred to the random stimuli of all the sensory inputs appear as higher than in the control group.
\begin{figure}[htbp]
    \begin{subfigure}[b]{0.5\textwidth}
        \centering
        \includegraphics[width=\textwidth]{bar_plots/plotbar_visual_s.png}
        \caption{Results of behavioral question in the visual condition in stroke population}
        \label{fig: bar_visual_stroke} 
    \end{subfigure} 
    \begin{subfigure}[b]{0.5\textwidth}
        \centering
        \includegraphics[width=\textwidth]{bar_plots/plotbar_audio_s.png}
        \caption{Results of behavioral question in the audio condition in stroke population}
        \label{fig: bar_audio_stroke} 
    \end{subfigure} 
    \begin{subfigure}[b]{0.5\textwidth}
        \centering
        \includegraphics[width=\textwidth]{bar_plots/plotbar_audiovisual_s.png}
        \caption{Results of behavioral question in the audiovisual condition in stroke population}
        \label{fig: bar_audiovisual_stroke} 
    \end{subfigure} 
\end{figure}

To confirm such difference between the two population groups, we decide to create a plot to visually compare them and to statistically measure their comparison. In Figure \ref{fig: mean_population_condition} we can recognize the distinction between the answers given by the stroke population and the control one to the random stimuli, and almost identical results for what concerns the rhythmic stimuli. If we then look at the statistical results (Figure \ref{fig: significance_total_mean_pop}) we can detect that the only significant result for the difference between the two population groups can be found for the answers related to the random visual inputs. 
\begin{figure}[htbp]
    \centering
    \includegraphics[width=0.5\textwidth]{bar_plots/mean stroke and control.png}
    \caption{Mean of the results of all behavioral question divided by population groups (CG = control group; SG = stroke group)}
    \label{fig: mean_population_condition} 
\end{figure} 

Later, we decided to correlate the most significant rhythmic \footnote{The choice of taking into account only the question in the rhythmic condition was made after seeing the results related to the difference between the two conditions: demonstrating that the rhythmic one got always higher values than the random.} question (q2) to the mean power activity of the Occipital, Temporal and Sensorimotor ROIs, dividing them always by population group. \\
We can examine the obtained results in Tables \ref{fig: correlation values q2: control} and \ref{fig: correlation values q2: stroke}. In the results for the control group we can observe that no correlation was found between the question related to the various sensory inputs and the power brain activity. Contrarily, looking at the results for the stroke population we can find a significant positive correlation between the values assigned to the behavioral question related to the audio stimuli and the neural activation in the Temporal ROI. Moving to the question related to the other sensory inputs (visual and audiovisual), no significant correlation with the mean power activity of the different ROIs could be found just as in the control group.

Finally, we also aimed to understand if a possible correlation between the level of physical activity of the participants and their correspondent mean power activation in the Sensorimotor area could be present. The outcomes of our analysis showed no correlation between the level of physical activity and the activation of the sensorimotor area related to any of the sensory inputs (Table \ref{fig: significance correlation activeq}). 

% At beginning of our EEG data analysis we plotted the averaged signal-to-noise ratio (SNR) to observe if the desired brain signal would have been distinguishable from the background EEG activity. We can observe our plots of the rhythmic and random conditions of the two population groups in Figure \ref{fig: snr_rhythmic: control} and \ref{fig: snr_rhythmic: stroke} for the rhythmic condition; and in Figure \ref{fig: snr_random: control} and \ref{fig: snr_random: stroke} for the random one. \\
% The graphs don't seem to differ much between control and stroke groups. The rhythmic conditions, in which the frequency has been precisely set at 2 Hz, show the highest peak at 2 Hz and its harmonic (4 Hz), while a flat SNR values for what concerns the other frequency can be spotted. An exception regards the frequency of 1 Hz were the SNR appears still lower than the interested peaks, but higher than the other non target frequencies. On the other hand, in the SNR related to the random conditions it is possible to detect higher peaks at frequency of 1 Hz and its harmonics 2, 3 and 4 together with some lower noise in all the other frequencies. 

% We proceeded to calculate how the power spectrum representing the brain activity of our participant would change across our three regions of interest (occipital, temporal and sensorimotor) during the presentation of the different stimuli in the two conditions, expecting to find a peak amplitude at 2 Hz in the rhythmic condition, concordant with the previous results. We calculated separately for the two different population groups the grand average of the neural activity in the ROIs related to the different stimuli and the two separate conditions. \\
% Giving a general look to the plots of all the ROIs (which can be found starting from Figure \ref{fig: occipital ROI: control}) we can see a clear activation peak at 2 Hz in the rhythmic conditions and not in random ones for both of the population groups, as expected. However, it is also noticeable that the peaks found in all the ROIs power spectrum of the stroke group appear as lower than in the control group, denoting a weaker brain activation. \\
% For what concerns the stimuli in the random conditions the results appear to be almost the same in all the ROIs: the higher power activity can be observed at 1Hz and its harmonics even if with a lower value than the peaks at 2 Hz in the rhythmic conditions.

% Observing the plots for the specific ROIs, we can spot in the occipital ROI (Figures \ref{fig: occipital ROI: control} and \ref{fig: 3D topographies stroke group}) it is possible to spot the peak at 2 Hz clearly in the rhythmic condition of the visual and audiovisual sensory inputs, whilst in the temporal ROI (Figures \ref{fig: temporal ROI: control} and \ref{fig: temporal ROI: stroke}) it can mainly be observed it in the audio and audiovisual rhythmic conditions. Considering the last ROI, the sensorimotor area (Figures \ref{fig: sensorimotor ROI: control} and \ref{fig: sensorimotor ROI: stroke}), the activity peak at 2 Hz can only be noticed in the audio and audiovisual rhythmic conditions; in the visual one, a higher brain activity is spotted only in the control group at 1 Hz, similarly to its activity power spectrum in the temporal ROI.

\chapter{Discussion}
The aim of the present study was to explore the brain dynamics supporting rhythmic sensorimotor synchronization when coupling auditory and visual inputs related to walking movement comparing healthy and stroke population, using an EEG frequency-tagging approach. \\
In the present study we conducted a mix-model experimental design presenting both auditory (footstep sound) and visual (walking point-light figure) inputs at the frequency rate of 2 Hz in a rhythmic or random sequence, in three different blocks (auditory, visual and audiovisual). We chose to adopt the EEG technique with a frequency tagging approach, which has already given significative results to explore sensorimotor synchronization in both healthy \parencite{Cracco_2022} and clinical populations, in particular with stroke patients \parencite{Nozaradan_2017}. \\
Additionally, we decided to insert in the experiment some behavioral questions to inspect how the participants felt towards the stimuli presented, we aimed to know if they could feel relaxed or involved with the movement produced by sensory inputs. The results collected from these questions would've given us a clue on why brain activation would differ between distinct stimuli or population groups. \\
All the EEG and behavioral results were compared between healthy subjects and patients to investigate the possible differences between them and the explanation standing behind them.

As we commented before, in a previous study the authors have found that when presenting different sensory inputs (auditory, visual, audiovisual) related to the walking movement in rhythmic sequences at 2 Hz frequency rate, compared with 1 and 3.6 Hz, there was a higher synchronization with the sensorimotor cortex, with audio and audiovisual sensory inputs.\\
In the current study, we confirm the previous results in healthy subjects and in patients with stroke. Regarding the synchronization with the sensorimotor cortex in the patients with stroke group for the auditory and audiovisual conditions, we observed a lateralization of the mean power activity to the left hemisphere. This can be due to the fact that there were more patients presenting the brain injury in the left hemisphere, compared to those who presented the injury in the right hemisphere (Table \ref{fig: Database stroke}). 

%% Auditory and audiovisual stimuli induce higher synchronization with the sensorimotor cortex.
One of the main findings concerns higher synchronization effects in response to rhythmic auditory and the audiovisual stimuli, whereas not in response to the visual one. The results of the auditory input depict the synchronization effect at the temporal and sensorimotor cortex; while we can observe that the audiovisual inputs (which combined footsteps sounds and the walking point-light figure) showed the activation of the occipital cortex, together with the temporal and sensorimotor. Finally, in the visual input, only the activation of the Occipital cortex could be found. These results were found in both healthy and stroke population (Figures \ref{fig: 3D topographies control group} and \ref{fig: 3D topographies stroke group}).

The crucial result of the induced higher synchronization effects within the sensorimotor cortex in the auditory input (footsteps sounds) compared to the visual input (walking point-light figure) can be explained through different previous findings. \\
The auditory cortices can interact with subcortical areas that deal with the temporal processing, generating an internal periodic pulse-like beat when listening to auditory rhythms \parencite{Nozaradan_2017}. Sound entrains humans to move: neuroimaging studies have shown that rhythmic sequences activate motor cortical areas \parencite{Grahn_2007}. \\
The difference between the synchronization in auditory and visual inputs can also be justified by the diverse role of attention in the auditory and visual system. In the last one it has been shown the key role of selective attention in visual search \parencite{Keshvari_2016}: there are elements that enhance this process, while others, such as biological motion and visual rhythms, that don't \parencite{Wolfe_2017}. Regarding this aspect, we can assume that it could be an explanation on why our visual sensory input (point-light figure) didn't enhance selective attention and, consequently, not induced sensorimotor synchronization as in the auditory inputs. \\
These outcomes confirm earlier studies stating that the auditory stimuli enhance the temporal lobe and function as the best activator for sensorimotor entrainment \parencite{Thaut_1999}. On the opposite hand, we saw that the visual inputs don't seem to elicit neural entrapment, but that they only provoke the activation of the occipital lobe \parencite{Nehmad_1998}, responsible for sight. 

Furthermore, we found that the EEG power amplitude recorded in stroke patients was weaker compared to the control group, with a left-lateralized activation. The topographies of mean power activity in patients revealed the same activated ROIs across sensory inputs as those identified in the control group, but with an overall reduction. This weaker activation, particularly in the sensorimotor cortex, temporal, and occipital areas, is more evident when comparing the rhythmic and random conditions. While the stroke group still showed stronger activation in response to rhythmic stimuli, the difference between conditions highlights how diminished the activation of these ROIs truly is. This result confirms the outcomes found by another study with stroke patients and EEG frequency tagging approach \parencite{Nozaradan_2017}. Furthermore, research has showed that poor motor function is related with reduced sensory pathway information leading to a diminished amplitude activity power \parencite{Campfens_2015}. \\
Another interesting distinction between the control group and the stroke patient is the lateralized activation on the left hemisphere that can be noticed in the activation related to all the sensory inputs in the rhythmic condition and with a weaker amplitude in the random one. This could be explained by the region of the injuries of our participants: the majority of them \footnote{13 subjects over 21 suffered from the injury on the left hemisphere, while one subject had both hemispheres lesioned \ref{fig: Database stroke}.} presented a lesion in the left hemispheres of the brain. An EEG study with chronic stroke participants has demonstrated that the lesion location actually influences the EEG pattern \parencite{Park_2016}, which can explain our findings. 

%% here preference for rhythmic vs random
Another important finding of this study is the higher sensorimotor synchronization in the rhythmic condition rather than in the random one. The EEG activity response to all the sensory inputs presented in a rhythmic or random sequence confirms the outcome of previous results, (e.g. Matamala-Gomez et al. in preparation; \cite{Haegens_2018}) showing a higher power amplitude in each region of interest for the stimuli in the rhythmic condition (see Topographies \ref{fig: 3D topographies control group}, \ref{fig: 3D topographies stroke group}). This can be explained by the need of brain oscillation to be synchronized to a repeated rhythm in order to enhance neural entrainment \parencite{Rosso_2023}. \\
Moreover, as mentioned before, attention plays a key role for sensory integration: it has been observed that the perception of incoherent inputs can initially be represented by a unified stimulus \parencite{Bergam_1990} and, consequently be ignored by the attentive system. This can disclose the higher synchronization with the sensory inputs repeated in the rhythmic sequence compared to the random one. 

%% Rhythmic presentation of the sensory input induces a higher sense of synchrony perception with the presented sensory inputs.     
% behavioral questions
In addition to those findings, the analysis of the behavioral questions showed significantly higher values for the rhythmic condition compared to the random one for every sensory input in both population groups, meaning that the rhythmic presentation of the sensory input induces a higher sense of synchrony perception with the presented sensory inputs. \\
From the obtained answers we could conclude that the participants perceived the rhythmic stimuli as enjoyable and fluid, while the opposite outcomes were found for the random stimuli. 

This corresponds to the higher neural entrainment found in the rhythmic condition of every stimulus compared to the almost none activation of the different ROIs in the random ones. The correlation between the most meaningful question and the ROIs' mean amplitude power didn't produce significant results, except for the temporal lobe amplitude related to the audio stimulus in the stroke population. \\
This last result is interesting and coherent with our expectation: higher positive feelings and involvement with the inputs are correlated to higher power amplitude in the regions of interest related to that particular stimulus. In contrast, the lack of correlation found between the other brain regions and sensory inputs could be explained by the use of not optimal statical measure or the nature of the data that was used.\\
Another important result obtained from the behavioral questions was the trend towards significance found for higher Likert values assigned to the questions corresponding to the random condition stimuli in the stroke group compared to the control group. This could be explained by the similarity of the hemiplegic walk of the stroke patient with the asynchronous movement represented with the random footsteps sounds and walk of the point-light figure. \\
In relation to the perception of the sensory inputs by the stroke group, registered both by EEG and the behavioral questions, we have to report that it may be influenced by sensory impairments suffered by the participants. In fact, up to 70\% of the stroke survivors can have impaired perception of different sensory stimuli, which may lead to misinterpretation of sensory information \parencite{Hazelton_2022} and consequent lower brain activation. 

Lastly, we report that we haven't found any significant correlation between the level of physical activity of the participants and the sensorimotor activation. 

To gain deeper insight into the results, especially the ones related to the stroke group, some advanced statistical analysis should be made, taking in consideration the different lesions' location, including the brain areas affected. We should see how patients with diverse injuries' location, differentiated by cortical and subcortical lesions, would integrate sensory stimuli and enhance neural entrainment. As mentioned before, it has been proven that the location of the lesion can influence EEG results, so a meticulous differentiation by this parameter, should be made. \\
Furthermore, more advance statistical analysis could be made for both EEG and behavioral data.

\chapter{Conclusion}
The present study provides insights on brain dynamics underlying neural entrainment and sensorimotor integration through the novel approach of frequency tagging using electroencephalogram, in different groups of individuals. The responses to the distinct sensory inputs were delineated by a diminished neural activation and a lateralized activation in patients who suffered from a stroke, compared to the healthy group. Additionally, the neural tracking in both populations showed enhanced results for sensory stimuli repeated in a rhythmic sequence rather than a random one, emphasizing the sensorimotor synchronization to periodic beat frequency. Such findings were also supported by the results of the behavioral questions which highlighted the positive feelings of fluidity and involvement with the movement in the rhythmic sequence, whereas these feelings were absent in the random one. \\
Above all, the results of the present study constitute important progress in the knowledge on the role of the cortical cortex on the integration of auditory and visual walking inputs in patients with stroke; giving insights on the possibility of employing the same sensory inputs used in the experiment, as tool for gait rehabilitation. Finding strategies to reacquire walking ability and consequently more independence from others, is crucial for patients both on a personal and social level. In order to get most appropriate results, that can later be used to adapt sensory inputs specifically for diverse lesion's location, further research taking into account this crucial variable is necessary.

\begin{appendices}
  \fancyhead[L]{APPENDICES}
\addcontentsline{toc}{chapter}{Appendix}
\chapter*{Appendix: Figures}
\section*{Experimental material}
\begin{figure}[ht]
    \centering
    \includegraphics[width=0.8\textwidth]{appendix/met_values.png}
    \caption{MET value labels \parencite{Bonn_2012}}
    \label{fig: met_values}
\end{figure}
\begin{figure}[ht]
    \centering
    \includegraphics[width=0.95\textwidth]{appendix/database.png}
    \caption{Participants database with results}
    \label{fig: database}
\end{figure}
\begin{figure}[ht]
    \centering
    \includegraphics[width=0.70\textwidth]{appendix/questions.png}
    \caption{Participants database with results}
    \label{fig: Behavioral questions}
\end{figure}

\clearpage
\section*{Results}
\subsection*{Activity power spectrum}
\begin{figure}[H]
    \centering
    \includegraphics[width=0.85\textwidth]{healthy_images/allROI_graph.png}
    \caption{Activity peaks taking all the ROIs in consideration}
    \label{fig: allROI}
\end{figure}
\begin{figure}[H]
    \centering
    \includegraphics[width=0.85\textwidth]{healthy_images/occipitalROI_graph.png}
    \caption{Activity peak in the occipital ROI}
    \label{fig: occipital ROI}
\end{figure}
\begin{figure}[H]
    \centering
    \includegraphics[width=0.85\textwidth]{healthy_images/temporalROI_graph.png}
    \caption{Activity peak in the temporal ROI}
    \label{fig: temporal ROI}
\end{figure}
\begin{figure}[H]
    \centering
    \includegraphics[width=0.85\textwidth]{healthy_images/sensorimotorROI_graph.png}
    \caption{Activity peak in the sensorimotor ROI}
    \label{fig: sensorimotor ROI} 
\end{figure} 

\clearpage
\subsection*{Topographies}
\begin{figure}[htbp]
    \centering
    \includegraphics[width=0.85\textwidth]{healthy_images/topo.png}
    \caption{Topographies related to the activity in the control group}
    \label{fig: topographies control group}
\end{figure}
\begin{figure}[htbp]
    \centering
    \includegraphics[width=0.85\textwidth]{stroke_images/topographies.png}
    \caption{Topographies related to the activity in stroke population}
    \label{fig: topographies stroke group}
\end{figure}
\begin{figure}[htbp]
    \centering
    \includegraphics[width=0.85\textwidth]{healthy_images/3d_topo.png}
    \caption{3D topographies related to the activity in the control group}
    \label{fig: 3D topographies control group}   
\end{figure} 
\begin{figure}[htbp]
    \centering
    \includegraphics[width=0.85\textwidth]{stroke_images/3d_topographies.png}
    \caption{3D topographies related to the activity in stroke population}
    \label{fig: 3D topographies stroke group}   
\end{figure} 

\clearpage
\subsection*{Bar plots related to behavioral questions}
% \subsubsection*{Total population}
% \begin{figure}[H]
%     \centering
%     \includegraphics[width=0.85\textwidth]{bar_plots/bar_visual_t.png}
%     \caption{Results of behavioral question in the visual condition in the total population}
%     \label{fig: bar_visual_Total} 
% \end{figure}
% \begin{figure}[H]
%     \centering
%     \includegraphics[width=0.85\textwidth]{bar_plots/bar_audio_t.png}
%     \caption{Results of behavioral question in the audio condition in the total population}
%     \label{fig: bar_audio_Total} 
% \end{figure} 
% \begin{figure}[H]
%     \centering
%     \includegraphics[width=0.85\textwidth]{bar_plots/bar_audiovisual_t.png}
%     \caption{Results of behavioral question in the audiovisual condition in the total population}
%     \label{fig: bar_audiovisual_Total} 
% \end{figure}  
% \begin{figure}[H]
%     \centering
%     \includegraphics[width=0.55\textwidth]{significance_tables/total_pop.png}
%     \caption{Significance of the results of behavioral questions in the total population, using Wilcoxon Signed-Rank Test (\textit{W})}
%     \label{fig: significance_total_pop} 
% \end{figure} 

\subsubsection*{Control group}
\begin{figure}[H]
    \centering
    \includegraphics[width=0.85\textwidth]{bar_plots/bar_visual_h.png}
    \caption{Results of behavioral question in the visual condition in the control group}
    \label{fig: bar_control_Total} 
\end{figure} 
\begin{figure}[H]
    \centering
    \includegraphics[width=0.85\textwidth]{bar_plots/bar_audio_h.png}
    \caption{Results of behavioral question in the audio condition in the control group}
    \label{fig: bar_audio_control} 
\end{figure} 
\begin{figure}[H]
    \centering
    \includegraphics[width=0.85\textwidth]{bar_plots/bar_audiovisual_h.png}
    \caption{Results of behavioral question in the audiovisual condition in the control group}
    \label{fig: bar_audiovisual_control} 
\end{figure} 
\begin{figure}[H]
    \centering
    \includegraphics[width=0.55\textwidth]{significance_tables/control_group.png}
    \caption{Significance of the results of behavioral questions in the control group, using Wilcoxon Signed-Rank Test (\textit{W})}
    \label{fig: significance_control_pop} 
\end{figure} 

\subsubsection*{Stroke group}
\begin{figure}[H]
    \centering
    \includegraphics[width=0.85\textwidth]{bar_plots/bar_visual_s.png}
    \caption{Results of behavioral question in the visual condition in stroke population}
    \label{fig: bar_visual_stroke} 
\end{figure} 
\begin{figure}[H]
    \centering
    \includegraphics[width=0.85\textwidth]{bar_plots/bar_audio_s.png}
    \caption{Results of behavioral question in the audio condition in stroke population}
    \label{fig: bar_audio_stroke} 
\end{figure} 
\begin{figure}[H]
    \centering
    \includegraphics[width=0.85\textwidth]{bar_plots/bar_audvis_s.png}
    \caption{Results of behavioral question in the audiovisual condition in stroke population}
    \label{fig: bar_audiovisual_stroke} 
\end{figure} 
\begin{figure}[H]
    \centering
    \includegraphics[width=0.60\textwidth]{significance_tables/stroke_group.png}
    \caption{Significance of the results of behavioral questions in the stroke group, using Wilcoxon Signed-Rank Test (\textit{W})}
    \label{fig: significance_stroke_pop} 
\end{figure} 

\subsubsection*{Total results mean}
% \begin{figure}[H]
%     \centering
%     \includegraphics[width=0.85\textwidth]{bar_plots/mean_questions.png}
%     \caption{Mean of the results of all behavioral question in both population groups}
%     \label{fig: mean_total} 
% \end{figure} 
% \begin{figure}[H]
%     \centering
%     \includegraphics[width=0.65\textwidth]{significance_tables/tot_mean.png}
%     \caption{Significance of the results of all the behavioral questions, using Wilcoxon Signed-Rank Test (\textit{W}) or Paired Samples t-Test (\textit{t})}
%     \label{fig: significance_tot_mean} 
% \end{figure} 
% \clearpage
\begin{figure}[H]
    \centering
    \includegraphics[width=0.85\textwidth]{bar_plots/mean stroke and control.png}
    \caption{Mean of the results of all behavioral question divided by population groups}
    \label{fig: mean_population_condition} 
\end{figure} 
\begin{figure}[H]
    \centering
    \includegraphics[width=0.70\textwidth]{significance_tables/tot_mean_pop.png}
    \caption{Significance of the results of all behavioral questions, comparing different population groups, using Wilcoxon Signed-Rank Test (\textit{W}) or Paired Samples t-Test (\textit{t})}
    \label{fig: significance_total_mean_pop} 
\end{figure} 

\clearpage
\subsection*{Correlation between physical activity and brain activation}
\begin{figure}[H]
    \centering
    \includegraphics[width=0.75\textwidth]{correlations/correlation_healthy.png}
    \caption{Correlation between activity and neural activation in the control group}
    \label{fig: correlation control} 
\end{figure}
\begin{figure}[H]
    \centering
    \includegraphics[width=0.75\textwidth]{correlations/correlation_stroke.png}
    \caption{Correlation between activity and neural activation in the stroke population}
    \label{fig: correlation stroke} 
\end{figure}
\begin{figure}[H]
    \centering
    \includegraphics[width=0.75\textwidth]{correlations/corr_values_activeq.png}
    \caption{Correlation values: Pearson correlation coefficient \textit{r} and \textit{p} value}
    \label{fig: correlation values} 
\end{figure}

\clearpage
\subsection*{Correlation between brain activation and behavioral questions}
\begin{figure}[H]
    \centering
    \includegraphics[width=0.75\textwidth]{correlations/corr_na_q_control.png}
    \caption{Correlation between neural activation and behavioral questions in the control group}
    \label{fig: correlation control questions} 
\end{figure}
\begin{figure}[H]
    \centering
    \includegraphics[width=0.75\textwidth]{correlations/corr_na_q_stroke.png}
    \caption{Correlation between neural activation and behavioral questions in the stroke population}
    \label{fig: correlation stroke questions} 
\end{figure}
\begin{figure}[H]
    \centering
    \includegraphics[width=0.75\textwidth]{correlations/corr_values_na_q.png}
    \caption{Correlation values: Pearson correlation coefficient \textit{r} and \textit{p} value}
    \label{fig: correlation values questions} 
\end{figure}
\end{appendices}

\printbibliography[heading=bibintoc]

\end{document}