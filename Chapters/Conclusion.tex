\chapter{Conclusion}
The present study provides insights on brain dynamics underlying neural entrainment and sensorimotor synchronization through the frequency-tagging approach with electroencephalogram, in different groups of individuals. The neural tracking in both populations showed higher sensorimotor synchronization in auditory and audiovisual stimuli rather than in the visual one. Further, sensorimotor synchronization, as well as temporal and occipital activation, were delineated with sensory stimuli repeated in a rhythmic sequence rather than a random one, emphasizing the sensorimotor synchronization to periodic beat frequency. Such findings were also supported by the results of the behavioral questions which highlighted the positive feelings of fluidity and involvement with the movement in the rhythmic sequence, whereas these feelings were absent in the random condition. The results for the stroke population were characterized by a diminished and lateralized neural activation compared to the healthy group. \\
Above all, the results of the present study constitute important progress in the knowledge on the role of the cortical cortex on the integration of auditory and visual walking inputs in patients with stroke. The results gave insights on the possibility of employing the same sensory inputs used in the experiment as tool for gait rehabilitation, where sensory feedback have already been proven to be beneficial to enhance motor response \parencite{Bolognini_2016}. Finding strategies to reacquire walking ability and independence from others is crucial for patients. \\
In order to get most appropriate results, that can later be used to adapt sensory inputs specifically for diverse lesion's location, further analysis, taking into account this crucial variable is necessary.

% Other studies have taken into consideration the investigation of sensory integration and motor activation in healthy control subjects (e.g. \parencite{Nozaradan_2011}, \parencite{Nozaradan_2014}), but not many have researched this phenomenon at the neural level in clinical population. 
% Besides, studies in post-stroke walking rehabilitation have demonstrated that sensory (visual, audio and audiovisual) feedback result beneficial to reacquire walking ability. Multisensory integration, which appears to be mostly integrated even in stroke patients, improves enhance motor responses: it helps redirect walking, and rhythmically coordinating action with predictable external stimulus . As for healthy subjects, auditory stimuli have been found to be crucial to activate neural entrainment, especially using movement sonification, i.e. the mapping of movement signals into sound.

% In conclusion, this study provides evidence for the role of attention to auditory input delivered in rhythmic walking sequences in synchronizing auditory areas with the sensorimotor cortex. Specifically, footsteps sound presented at 2 Hz induced a peak amplitude for synchronization with the sensorimotor cortex. Further, when footstep sounds were combined with a walking point-light figure (presented at the same frequency rate), they likely enhanced attention and induced higher synchronization with the sensorimotor cortex. The results from this study may be relevant for designing sensory learning and multisensory integration trainings for motor rehabilitation, based on the neurobiology of rhythmic motor entrainment (Molinari et al. 2003), particularly for the rehabilitation of rhythmic movements such as walking. 