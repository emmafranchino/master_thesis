\chapter{Conclusion}
The present study provides insights on brain dynamics underlying neural entrainment and sensorimotor integration through the novel approach of frequency tagging using electroencephalogram, in different groups of individuals. The responses to the distinct sensory inputs were delineated by a diminished neural activation and a lateralized activation in patients who suffered from a stroke, compared to the healthy group. Additionally, the neural tracking in both populations showed enhanced results for sensory stimuli repeated in a rhythmic sequence rather than a random one, emphasizing the sensorimotor synchronization to periodic beat frequency. Such findings were also supported by the results of the behavioral questions which highlighted the positive feelings of fluidity and involvement with the movement in the rhythmic sequence, whereas these feelings were absent in the random one. \\
Above all, the results of the present study constitute important progress in the knowledge on the role of the cortical cortex on the integration of auditory and visual walking inputs in patients with stroke; giving insights on the possibility of employing the same sensory inputs used in the experiment, as tool for gait rehabilitation. Finding strategies to reacquire walking ability and consequently more independence from others, is crucial for patients both on a personal and social level. In order to get most appropriate results, that can later be used to adapt sensory inputs specifically for diverse lesion's location, further research taking into account this crucial variable is necessary.