\chapter{Discussion}
The aim of the present study was to explore the brain dynamics supporting rhythmic sensorimotor synchronization when coupling auditory and visual inputs related to walking movement comparing healthy and stroke population, using an EEG frequency-tagging approach. 
In the present study we designed a task in which we presented an auditory (footstep sound) and visual (walking point-light figure) inputs at the frequency rate of 2 Hz in a rhythmic or random sequence, in three different blocks (auditory, visual and audiovisual). We chose to adopt the EEG technique with a frequency tagging approach, which has already given significative results to explore sensorimotor synchronization in both healthy \parencite{Cracco_2022} and clinical populations, in particular with stroke patients \parencite{Nozaradan_2017}. 
Additionally, we decided to insert in the experiment some subjective report questions to inspect how the participants felt towards the stimuli presented, we aimed to know if they could feel relaxed or involved with the movement produced by sensory inputs. The results collected from these questions would have given us a clue on why brain activation would differ between distinct stimuli or population groups. 
All the EEG and subjective report results were compared between healthy subjects and patients to investigate the possible differences between them and the explanation standing behind them.
As we commented before, in a previous study the authors have found that when presenting different sensory inputs (auditory, visual, audiovisual) related to the walking movement in rhythmic sequences at 2 Hz frequency rate, compared with 1 and 3.6 Hz, there was a higher synchronization with the sensorimotor cortex, with audio and audiovisual sensory inputs.
In the current study, we confirmed the previous results in healthy subjects and in patients with stroke. 
% Regarding the synchronization with the sensorimotor cortex in the patients with stroke group for the auditory and audiovisual conditions, we observed a lateralization of the mean power activity to the left hemisphere. This can be due to the fact that there were more patients presenting the brain injury in the left hemisphere, compared to those who presented the injury in the right hemisphere (Table \ref{fig: Database stroke}). 

%% Auditory and audiovisual stimuli induce higher synchronization with the sensorimotor cortex.
One of the main findings concerns higher synchronization effects in response to rhythmic auditory and the audiovisual stimuli, whereas not in response to the visual one. The results of the auditory input depict the synchronization effect at the temporal and sensorimotor cortex; while we can observe that the audiovisual inputs (which combined footsteps sounds and the walking point-light figure) showed the activation of the occipital cortex, together with the temporal and sensorimotor. Finally, in the visual input, only the activation of the occipital cortex could be found. These results were found in both healthy and stroke population (Figures \ref{fig: 3D topographies control group} and \ref{fig: 3D topographies stroke group}).
The crucial result of the induced higher synchronization effects within the sensorimotor cortex in the auditory input (footsteps sounds) compared to the visual input (walking point-light figure) can be explained through different previous findings. 
The auditory cortices can interact with subcortical areas that deal with the temporal processing, generating an internal periodic pulse-like beat when listening to auditory rhythms \parencite{Nozaradan_2017}. Sound entrains humans to move: neuroimaging studies have shown that rhythmic sequences activate motor cortical areas \parencite{Grahn_2007}. 
The difference between the synchronization in auditory and visual inputs can also be justified by the diverse role of attention in the auditory and visual system. In the last one it has been shown the key role of selective attention in visual search \parencite{Keshvari_2016}: there are elements that enhance this process, while others, such as biological motion and visual rhythms, that don't \parencite{Wolfe_2017}. Regarding this aspect, we can assume that it could be an explanation on why our visual sensory input (point-light figure) did not enhance selective attention and, consequently, not induced sensorimotor synchronization as in the auditory inputs. 
These outcomes confirm earlier studies stating that the auditory stimuli enhance the temporal lobe and function as the best activator for sensorimotor entrainment \parencite{Thaut_1999}. On the opposite hand, we saw that the visual inputs did not seem to elicit neural entrapment, but that they only provoke the activation of the occipital lobe \parencite{Nehmad_1998}, responsible for sight. 

Furthermore, we found that the EEG power amplitude recorded in stroke patients was weaker compared to the control group. The topographies of mean power activity in patients revealed the same activated ROIs across sensory inputs as those identified in the control group, but with an overall reduction. This weaker activation, particularly in the sensorimotor cortex, temporal, and occipital areas, is more evident when comparing the rhythmic and random conditions. While the stroke group still showed stronger activation in response to rhythmic stimuli, the difference between conditions highlights how diminished the activation of these ROIs truly is. This result confirms the outcomes found by another study with stroke patients and EEG frequency tagging approach \parencite{Nozaradan_2017}. Furthermore, research has showed that poor motor function is related with reduced sensory pathway information leading to a diminished amplitude activity power \parencite{Campfens_2015}. 
% Another interesting distinction between the control group and the stroke patient is the lateralized activation on the left hemisphere that can be noticed in the activation related to all the sensory inputs in the rhythmic condition and with a weaker amplitude in the random one. This could be explained by the region of the injuries of our participants: the majority of them \footnote{13 subjects over 21 suffered from the injury on the left hemisphere, while one subject had both hemispheres lesioned \ref{fig: Database stroke}.} presented a lesion in the left hemispheres of the brain. An EEG study with chronic stroke participants has demonstrated that the lesion location actually influences the EEG pattern \parencite{Park_2016}, which can explain our findings. 

%% here preference for rhythmic vs random
Another important finding of this study is the higher sensorimotor synchronization in the rhythmic condition rather than in the random one. The EEG activity response to all the sensory inputs presented in a rhythmic or random sequence confirms the outcome of previous results, (e.g. Matamala-Gomez et al. in preparation; \cite{Haegens_2018}) showing a higher power amplitude in each region of interest for the stimuli in the rhythmic condition (see Topographies \ref{fig: 3D topographies control group}, \ref{fig: 3D topographies stroke group}). This can be explained by the need of brain oscillation to be synchronized to a repeated rhythm in order to enhance neural entrainment \parencite{Rosso_2023}. 
Moreover, as mentioned before, attention plays a key role for sensory integration: it has been observed that the perception of incoherent inputs can initially be represented by a unified stimulus \parencite{Bergam_1990} and, consequently be ignored by the attentive system. This can disclose the higher synchronization with the sensory inputs repeated in the rhythmic sequence compared to the random one. 

%% Rhythmic presentation of the sensory input induces a higher sense of synchrony perception with the presented sensory inputs.     
% subjective report questions
In addition to those findings, the analysis of the subjective report questions showed significantly higher values for the rhythmic sequences compared to the random sequences after the presentation of each sensory input in both healthy subjects and patients with stroke. From this result, we can hypothesize that during the presentation of the rhythmic sequences of the sensory input, the participants felt synchronized and more involved with the frequency rate of the sensory input (2Hz). This result is in line with previous studies \parencite{Nozaradan_2016}. 
From the reported results on the subjective questions, we also can state that in both groups, participants perceived the rhythmic stimuli as more enjoyable, and fluid compared to the random sequences. 

Moreover, from the subjective questions we also have shown that participant perceived rhythmic sequences the participants experienced a sense of agency (Q4: \textit{During the observation of the video, I felt like I was inside the body of the white dots (as if my own body was performing the movement)}, adapted for each sensory input) toward the observed point-light-figure walking movement in the visual and audiovisual experimental blocks and toward the footsteps sound in the auditory experimental block. This is an interesting result, as even we observed a coupling of the auditory input with the sensorimotor cortex when presenting the auditory input in rhythmic sequences, we did not observe the same effect with the visual input presented in rhythmic sequences. However, different studies have shown that the sense of agency toward an action is related to activation with motor and sensorimotor areas \parencite{Buhrmann_2017,Longo_2009}. Then, future studies can be directed in investigating whether providing a different visual input perspective or body morphology reproducing the walking movement at 2 Hz can synchronize with the sensorimotor cortex. 

Even if previous studies have discussed that up to 70\% of the stroke survivors can have impaired perception of different sensory stimuli, which may lead to misinterpretation of sensory information \parencite{Hazelton_2022} and consequent lower brain activation, in the present study we did not show any issue in the integration or interpretation of the provided sensory inputs (walking point-light-figure and footstep sounds). However, from the topographies presented in figure 3.8, we also observe a lower synchronization effect with the sensorimotor cortex in patients with stroke compared to the healthy control group.
In order to obtain a major knowledge about the possible differences in the synchronization of the different sensory inputs with the different brain ROIs between healthy and patients with stroke, further statistical analyses will be conducted. Further, we will investigate possible differences in sensory input integration in patients with cortical or subcortical injuries. It has been proven that the location of the lesion can influence EEG results \parencite{Clark_2015}, so a meticulous differentiation by this parameter, should be made. 

% This corresponds to the higher neural entrainment found in the rhythmic condition of every stimulus compared to the almost none activation of the different ROIs in the random ones. The correlation between the most meaningful question and the ROIs' mean amplitude power didn't produce significant results, except for the temporal lobe amplitude related to the audio stimulus in the stroke population. 
% This last result is interesting and coherent with our expectation: higher positive feelings and involvement with the inputs are correlated to higher power amplitude in the regions of interest related to that particular stimulus. In contrast, the lack of correlation found between the other brain regions and sensory inputs could be explained by the use of not optimal statical measure or the nature of the data that was used.
% Another important result obtained from the subjective report questions was the trend towards significance found for higher Likert values assigned to the questions corresponding to the random condition stimuli in the stroke group compared to the control group. This could be explained by the similarity of the hemiplegic walk of the stroke patient with the asynchronous movement represented with the random footsteps sounds and walk of the point-light figure. 
% In relation to the perception of the sensory inputs by the stroke group, registered both by EEG and the subjective report questions, we have to report that it may be influenced by sensory impairments suffered by the participants. In fact, up to 70\% of the stroke survivors can have impaired perception of different sensory stimuli, which may lead to misinterpretation of sensory information  and consequent lower brain activation. 

% Lastly, we report that we haven't found any significant correlation between the level of physical activity of the participants and the sensorimotor activation. 

% To gain deeper insight into the results, especially the ones related to the stroke group, some advanced statistical analysis should be made, taking in consideration the different lesions' location, including the brain areas affected. We should see how patients with diverse injuries' location, differentiated by cortical and subcortical lesions, would integrate sensory stimuli and enhance neural entrainment. As mentioned before, it has been proven that the location of the lesion can influence EEG results, so a meticulous differentiation by this parameter, should be made. 
% Furthermore, more advance statistical analysis could be made for both EEG and subjective report data.