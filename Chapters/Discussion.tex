\chapter{Discussion}
Discuss your results and their importance, for example, if you have worked with one language would they be applicable to the faculty of language in general? What do your results mean for the theory of language in general? Can they be used to improve speech therapy intervention? Which limitations did you find in your work? Sample size? Uncontrolled variables?

% The observation that target frequencies dominate the spectrum has been commonly taken as evidence for the underlying entrainment of neural oscillations (e.g., Nozaradan et al., 2011 ; Nozaradan et al., 2012 ; Lenc et al., 2018 ), which is not exempt from critiques

% These areas were chosen as ROIs because they were expected to be activated by the various stimuli: the visual stimuli should've activated the occipital ROI, the audio stimuli the temporal ROIs, the audiovisual both of these last ones; meanwhile the sensorimotor area was expected to activate when the neural entrainment would take place. 

%%% --- useful for the results --- %%%
% Neural oscillation signals obtained through electroencephalography (EEG) consistently reflect specific cognitive and sensory processes (Thut et al., 2012). 

% Additionally, across segments, the degree of phase-locking for frequency-specific EEG oscillations can be measured (Lachaux et al.,  1999; Roach and Mathalon, 2008). When data are summed across trials, the activity represents evoked and induced responses. To obtain a measure of induced activity, the evoked activity is subtracted from the summed activity. 

% At the basic level, neurons in the primary auditory cortex have a dual representation of the repetition rate of a stimulus, by discharge rate, and by synchronized firing patterns of a group of neurons (Bendor and Wang, 2007; Eggermont, 2001).
% The observation that target frequencies dominate the spectrum has been commonly taken as evidence for the underlying entrainment of neural oscillations (e.g., Nozaradan et al., 2011 ; Nozaradan et al., 2012 ; Lenc et al., 2018 ), which is not exempt from critiques

%%% --- poster Marta
% Frequency tagging measures a periodic change in voltage amplitude in the electrical activity recorded on the human scalp by using EEG when presenting stimuli repeated at a fixed rate1. Regarding auditory stimulation, some studies show a tonic synchronization response when using periodic auditory stimulation set at 2Hz2. Further, others found brain responses associated with biological motion perception by showing a point-light walker video moving at a pace of 2.4Hz3.
% Movement sonification implies the mapping of movement signals into sound, that can be used for motor rehabilitation4. To date, none have investigated frequency tagging responses and the neural correlates related to movement sonification techniques for motor rehabilitation.
% Aim of the study
% In this study we aimed to evaluate the impact of sonification on walking ability as indexed by neural markers and brain dynamics on the sensorimotor, the auditory, and the visual cortex using a frequency tagging approach in healthy subjects.


%% The idea of using auditory feedback in interactive systems has recently gained momentum in different research fields. In applications such as movement rehabilitation, sport training or product design, the use of auditory feedback can complement visual feedback. It reacts faster than the visual system and can continuously be delivered without constraining the movements. In particular, movement sonification systems appear promising for sensori-motor learning in providing users with auditory feedback of their own movements. Generally, sonification is defined as the use of non-speech audio to convey information (Kramer et al., 1999). Nevertheless, research on movement sonification for sensori-motor learning has been scattered in totally different research fields. On the one hand, most neuroscience and medical experiments have made use of very basic interactive systems, with little concern for sound design and the possible types of sonification. On the other hand, novel sound/music interactive technologies have been developed toward artistic practices, gaming or sound design, with little concern for sensori-motor learning.