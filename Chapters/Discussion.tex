\chapter{Discussion}
Previous research has demonstrated that sensory inputs repeated in a rhythmic sequence induce the neural entrainment, through a correct sensorimotor integration, activating motor responses. Sensorimotor synchronization has been examined with different sensory modalities such as visual, audio or multisensory stimuli. What has been found is a higher activation of sensorimotor cortex with auditory stimuli \parencite{Joris_2024}: the auditory cortices can interact with subcortical areas that deal with the temporal processing, generating an internal periodic pulse-like beat when listening to auditory rhythms \parencite{Nozaradan_2017}. Sound entrains humans to move: neuroimaging studies have shown that rhythmic sequences activate motor cortical areas \parencite{Grahn_2007}. \\
Furthermore, studies in post-stroke walking rehabilitation have demonstrated that sensory (visual, audio and audiovisual) feedback result beneficial to reacquire walking ability. Multisensory integration, which appears to be mostly integrated even in stroke patients, improves enhance motor responses: it helps redirect walking, and rhythmically coordinating action with predictable external stimulus \parencite{Bolognini_2016}. As for healthy subjects, auditory stimuli have been found to be crucial to activate neural entrainment, especially using movement sonification, i.e. the mapping of movement signals into sound.

Taking into account this experimental evidence, we aimed to explore the brain dynamics underlying sensorimotor integration and neural entrainment, generated by auditory and visual stimuli related to walking. We also intended to make a comparison between control healthy subjects and patients who suffered from a stroke. Other studies have taken into consideration the investigation of sensory integration and motor activation in healthy control subjects (e.g. \parencite{Nozaradan_2011}, \parencite{Nozaradan_2014}), but not many have researched this phenomenon at the neural level in clinical population. \\
The decision to explore these brain dynamics in patients who have experienced a stroke was made to observe how the lesioned brain integrates sensory stimuli, if it is able to synchronize correctly with the rhythm of the sensory inputs; if so, the auditory and visual inputs related to walking employed in the experiment, could be later used as therapeutic tools in post-stroke gait rehabilitation. 

In order to investigate the brain dynamics of neural entrainment, we chose to adopt the EEG technique with a frequency tagging approach, which has already given great results to explore sensorimotor sonification in both healthy \parencite{Cracco_2022} and clinical populations, in particular with stroke patients \parencite{Nozaradan_2017}. A study conducted prior to the present one, conducted with EEG frequency-tagging on healthy subjects demonstrated that the best inputs to activate a motor synchronization were the audio and audiovisual stimuli repeated in rhythmic sequence at the frequency of 2 Hz (Matamala et al. in preparation). Taking this result into consideration we decided to conduct a mix model study selecting the same sensory stimuli: we adopted footsteps sounds for the auditory input, a point-light figure walking for the visual one and a combination of the two for a multisensory one (referred to as the audiovisual stimuli). The inputs were set at the most adequate frequency of 2 Hz, with both random and rhythmic sequence, in order to see if the response to the different conditions would change through population groups. \\
Additionally, we decided to insert in the experiment some behavioral questions to inspect how the participants felt towards the stimuli presented, we aimed to know if they could feel relaxed or involved with the movement produced by sensory inputs. The results collected from these questions would've given us a clue on why brain activation would differ between distinct stimuli or population groups. \\
All the EEG and behavioral results were compared between healthy subjects and patients to investigate the possible differences between them and the explanation standing behind them. 

%% add citations
The EEG activity response to all the sensory inputs presented in a rhythmic or random sequence confirms the outcome of previous results, (e.g. Matamala et al. in preparation; \parencite{Haegens_2018}) showing a higher power amplitude in each region of interest for the stimuli in the rhythmic condition. This can be explained by the need of brain oscillation to be synchronized to a repeated rhythm in order to enhance neural entrainment \parencite{Rosso_2023}. \\
Looking at the brain activation of the healthy subjects the expected results can be detected: in auditory stimulus the activation of the temporal ROI and a strong sensorimotor arousal is present; in the audiovisual one the occipital area is added to the other activated ROIs; whilst in the visual input only the occipital area looks activated. These outcomes confirm earlier studies stating that the auditory stimuli enhance the temporal lobe and function as the best activator for sensorimotor entrainment \parencite{Thaut_1999}. On the opposite hand, we saw that the visual inputs don't seem to elicit neural entrapment, but that they only provoke the activation of the occipital lobe \parencite{Nehmad_1998}, responsible for sight. \\
The activation of the regions of interest in the different stimuli in the rhythmic conditions and not in the random one can be clearly seen in the topography of the difference between the two. 

The EEG power amplitude recorded in stroke patients was weaker compared to the control group, with a strong left-lateralized activation. The topographies of mean power activity in patients revealed the same ROIs across sensory inputs as those identified in the control group, but with an overall reduction. This weaker activation, particularly in the sensorimotor cortex, temporal, and occipital areas, is more evident when comparing the rhythmic and random conditions. While the stroke group still showed stronger activation in response to rhythmic stimuli, the difference between conditions highlights how diminished the activation of these ROIs truly is. This result confirms the outcomes found by another study with stroke patients and EEG frequency tagging approach \parencite{Nozaradan_2017}. \\
Another interesting distinction between the control group and the stroke patient is the lateralized activation on the left hemisphere that can be noticed in the activation related to all the sensory inputs in the rhythmic condition and with a weaker amplitude in the random one. This could be explained by the region of the injuries of our participants: the majority of them \footnote{13 subjects over 21 suffered from the injury on the left hemisphere, while one subject had both hemispheres lesioned.} presented a lesion in the left hemispheres of the brain. Furthermore, an EEG study with chronic stroke participants has demonstrated that the lesion location actually influences the EEG pattern \parencite{Park_2016}.

In addition to those findings, the analysis of the behavioral questions show significantly higher values for the rhythmic condition compared to the random one for every sensory input in both population groups, meaning that the participants felt involved in the movement displayed by the point-light figure and footsteps sounds. They perceived the rhythmic stimuli as enjoyable and fluid, while the opposite outcomes were found for the random stimuli. \\
This corresponds to the higher neural entrainment found in the rhythmic condition of every stimulus compared to the almost none activation of the different ROIs in the random ones. The correlation between the most meaningful question and the ROIs' mean amplitude power didn't produce significant results, except for the temporal lobe amplitude related to the audio stimulus in the stroke population. \\
This last result is interesting and coherent with our expectation: higher positive feelings and involvement with the inputs are correlated to higher power amplitude in the regions of interest related to that particular stimulus. In contrast, the lack of correlation found between the other brain regions and sensory inputs could be explained by the use of not optimal statical measure or the nature of the data that was used.\\
Another curios result related obtained from the behavioral questions was the higher Likert values assigned to the ones corresponding to the random condition stimuli, which appear as slightly higher in the stroke group rather than in the control one; with a trend toward significance in the random visual inputs. This could be explained by the similarity of the hemiplegic walk of the stroke patient with the asynchronous movement represented with the random footsteps sounds and walk of the point-light figure.   

In relation to the perception of the sensory inputs by the stroke group, registered both by EEG and the behavioral questions, we have to report that it may be influenced by sensory impairments suffered by the participants. In fact, up to 70\% of the stroke survivors may have impaired perception of different sensory stimuli, which may lead to misinterpretation of sensory information \parencite{Hazelton_2022} and consequent lower brain activation. \\
Furthermore, research has showed that poor motor function is related with reduced sensory pathway information leading to a diminished amplitude activity power \parencite{Campfens_2015}.

Lastly, we report that we haven't found any significant correlation between the level of physical activity of the participants and the sensorimotor activation. 

To gain deeper insight into the results, especially the ones related to the stroke group, some advanced statistical analysis should be made, taking in consideration the different lesions' location, including the brain areas affected. We should see how patients with diverse injuries, differentiated by cortical and subcortical lesions, would integrate sensory stimuli and enhance neural entrainment. As mentioned before, it has been proven that the location of the lesion can influence EEG results, so a meticulous differentiation by this parameter should be made. \\
Furthermore, more advance statistical analysis could be made for behavioral questions and their correlation with brain amplitude power. 