\chapter{Results}
\section{EEG results}
The outcome of our EEG analysis are descriptive EEG results, which will be deepened in later studies through statistical analysis.\\At beginning of our EEG data analysis we plotted the averaged signal-to-noise ratio (SNR) to observe if the desired brain signal would have been easy to distinguish from the background EEG activity. We can observe our plots of the random (Figures \ref{fig: snr_random: control}, \ref{fig: snr_random: stroke}) and rhythmic (Figures \ref{fig: snr_rhythmic: control} and \ref{fig: snr_rhythmic: stroke}) conditions of the two population groups. The graph don't seem to differ much between control and stroke groups: the rhythmic conditions show a high peak at 2 Hz and its harmonic (4 Hz), nonetheless also some lower peaks at the frequency of 1 and 3 Hz, combined with some general noise at the other intermediate frequencies. We could affirm that the peaks at 1 and 3Hz as well as the higher SNR in the other ones is produced by the random conditions, considering the (figure number) where high SNR can be noticed at frequency of 1 Hz and its harmonics 2, 3 and 4 together with noise in all the other frequencies. \\
On the other hand, if we consider the SNR of the rhythmic conditions, the ones that interests us the most, in which the frequency has been precisely set at 2 Hz, we can actually notice our highest peak at 2 Hz and its harmonic of 4 Hz, while a flat SNR values for what concerns the other frequency. An exception regards the frequency of 1 Hz were the SNR appears still lower than the interested peaks, but higher than the other non target frequencies. 

Knowing that the rhythmic conditions had a strong neural response at the desired frequencies of 2 Hz and harmonics, we proceeded to calculate how the power spectrum representing the brain activity of our participant would change across time and frequency of our stimuli. We used the grand average of all the participants' neural activity in relation to the different stimuli (audio, audiovisual, visual) and the two separate conditions. We made distinct plots for the activity of various sets of electrodes, representing our regions of interest (ROI): the first plot took into account all the electrodes, then the others were for the occipital, temporal and sensorimotor area. \\
These areas were chosen as ROIs because they were expected to be activated by the various stimuli: the visual stimuli should've activated the occipital ROI, the audio stimuli the temporal ROIs, the audiovisual both of these last ones; meanwhile the sensorimotor area was expected to activate when the neural entrainment would take place. 

In Figure \ref{fig: allROI} it is possible to spot the peak at 2 Hz clearly in the Audio and Audiovisual rhythmic conditions, while in the visual one a considerably high peak is also shown at 1 Hz, whilst a lower one can be found at the desired frequency. 
For what concerns the stimuli in the random conditions the results appear to be almost the same in all the ROIs including this first one with all the electrodes: the higher power activity can be observed at 1Hz and its harmonics even if with a lower value than the peaks at 2 Hz in the rhythmic conditions. \\
If we move into specific ROIs we can encounter the expected outcomes: in the occipital ROI \ref{fig: occipital ROI} the peak at 2 Hz in the rhythmic condition is primarily visible in the visual and audiovisual condition, whilst in the temporal ROI \ref{fig: temporal ROI} we can mainly observe it in the audio and audiovisual rhythmic conditions. Regarding the last ROI, the sensorimotor area \ref{fig: sensorimotor ROI}, the activity peak at 2 Hz can only be noticed in the audio and audiovisual rhythmic conditions; in the visual one, a higher brain activity is spotted only at 1 Hz, similarly to its activity power spectrum in the temporal ROI.

Having the desired graphs representing the activity power spectrum over frequencies, we proceeded to plot the two- and three-dimensional topographies concerning the brain activity just in the specific frequency of 2 Hz, related to the three different stimuli, for both the rhythmic and random conditions, adding the difference between these two. \\
In Figure \ref{fig: topographies control group} and Figure \ref{fig: 3D topographies control group} we can analyze the neural activity of the control group: a strong activation of the temporal and sensorimotor areas were observed both in the audio and audiovisual rhythmic conditions, with a broader spread of the activity in the surrounding areas in the audio condition and the supplementary activation of the occipital lobe in the during the audiovisual stimuli. For what concerns the brain activation during the rhythmic visual stimuli, it can be seen clearly only in the occipital area. \\
Moving to the topographies related to the random condition, the pattern of the brain activation appears as much weaker than the rhythmic condition, with a feeble arousal in the interested ROIs. \\
We can conclude that the frequency tagging effect is greater in the rhythmic conditions, with an activation of the ROIs that can be detected through the difference between the two conditions. 

For what concerns the stroke population the encountered results (Figure \ref{fig: topographies stroke group} and Figure \ref{fig: 3D topographies stroke group}) appear as slightly different from the control group. During the audio stimuli, the temporal and sensorimotor areas looked activated, even if not as much as in the control group. Furthermore, strong lateralized arousal on the left side of the brain can be denoted in both the topographies related to the audio and audiovisual stimuli. \\
On the other hand, the visual stimuli elicited a weaker activity in the stroke population compared to the control group, as well as a softer left sided activation than the audio and audiovisual ones. 




\section{Behavioral results}
Here we discuss the results related to the behavioral questions (bar charts, scatter plots and correlations) and the ones related to the activity scores (scatter plots and correlations).