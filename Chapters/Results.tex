\chapter{Results}
\section{EEG results}
The outcome of our EEG analysis are descriptive EEG results, which will be deepened later, through advanced statistical analysis.\\ 
At beginning of our EEG data analysis we plotted the averaged signal-to-noise ratio (SNR) to observe if the desired brain signal would have been distinguishable from the background EEG activity. We can observe our plots of the rhythmic and random conditions of the two population groups in Figure \ref{fig: snr_rhythmic: control} and \ref{fig: snr_rhythmic: stroke} for the rhythmic condition and in Figure \ref{fig: snr_random: control} and \ref{fig: snr_random: stroke} for the random one. \\
The graphs don't seem to differ much between control and stroke groups.  The rhythmic conditions, in which the frequency has been precisely set at 2 Hz, show the highest peak at 2 Hz and its harmonic (4 Hz), while a flat SNR values for what concerns the other frequency. An exception regards the frequency of 1 Hz were the SNR appears still lower than the interested peaks, but higher than the other non target frequencies. On the other hand, in the SNR related to the random conditions it is possible to detect higher peaks at frequency of 1 Hz and its harmonics 2, 3 and 4 together with some lower noise in all the other frequencies. 

We proceeded to calculate how the power spectrum representing the brain activity of our participant would change across our three regions of interest (occipital, temporal and sensorimotor) during the presentation of the different stimuli in the two conditions, expecting from the last results to find a peak amplitude at 2 Hz in the rhythmic condition. We calculated separately for the two different population groups the grand average of the neural activity in the ROIs related to the different stimuli and the two separate conditions. \\
Giving a general look to the plots of all the ROIs (that can be found starting from Figure \ref{fig: occipital ROI: control}) we can see a clear activation peak at 2 Hz in the rhythmic conditions and not in random ones for both of the population groups, as expected. However, it is also noticeable that the peaks found in all the ROIs power spectrum of the stroke group appear as lower than in the control group, denoting a weaker brain activation.  
% For what concerns the stimuli in the random conditions the results appear to be almost the same in all the ROIs including this first one with all the electrodes: the higher power activity can be observed at 1Hz and its harmonics even if with a lower value than the peaks at 2 Hz in the rhythmic conditions.

Observing the plots for the specific ROIs, we can spot in the occipital ROI (Figures \ref{fig: occipital ROI: control} and \ref{fig: 3D topographies stroke group}) it is possible to spot the peak at 2 Hz clearly in the rhythmic condition of the visual and audiovisual sensory inputs, whilst in the temporal ROI (Figures \ref{fig: temporal ROI: control} and \ref{fig: temporal ROI: stroke}) it can mainly be observed it in the audio and audiovisual rhythmic conditions. Considering the last ROI, the sensorimotor area (Figures \ref{fig: sensorimotor ROI: control} and \ref{fig: sensorimotor ROI: stroke}), the activity peak at 2 Hz can only be noticed in the audio and audiovisual rhythmic conditions; in the visual one, a higher brain activity is spotted only in the control group at 1 Hz, similarly to its activity power spectrum in the temporal ROI.

Having the desired graphs representing the activity power spectrum over frequencies, we proceeded to plot the two- and three-dimensional topographies concerning the brain activity of the different population groups, just in the specific frequency of 2 Hz, related to the three different stimuli, for both the rhythmic and random conditions, adding the difference between these two. \\
In Figure \ref{fig: topographies control group} and Figure \ref{fig: 3D topographies control group} we can analyze the neural activity of the control group: a strong activation of the temporal and sensorimotor areas were observed both in the audio and audiovisual rhythmic conditions, with a broader spread of the activity in the surrounding areas in the audio condition and the supplementary activation of the occipital lobe in the during the audiovisual stimuli. For what concerns the brain activation during the rhythmic visual stimuli, it can be seen clearly only in the occipital area. \\
Moving to the topographies related to the random condition, the pattern of the brain activation appears as much weaker than the rhythmic condition, with a feeble arousal in some different brain regions. \\
We can conclude that the frequency tagging effect is greater in the rhythmic conditions, with an activation of the ROIs that can be detected through the difference between the two conditions. 

For what concerns the stroke population the encountered results (Figure \ref{fig: topographies stroke group} and Figure \ref{fig: 3D topographies stroke group}) appear as slightly different from the control group. During the audio stimuli, the temporal and sensorimotor areas looked activated, even if not as much as in the control group. Furthermore, strong lateralized arousal on the left side of the brain can be denoted in both the topographies related to the audio and audiovisual stimuli. On the other hand, the visual stimuli elicited a weaker activity in the stroke population compared to the control group, as well as a softer left sided activation than the audio and audiovisual ones. \\
As in the control group, the random condition present a general weak brain activation, with the main difference of a noticeable arousal for what concerns the left side lateralization. \\
Finally, the difference between the two conditions confirms the activation of the diverse ROIs related to the corresponding sensory inputs, resulting more feeble than the ones found in the control group. 

\section{Behavioral results}
For the analysis of the behavioral results, we firstly made various bar plots divided for population groups and sensory inputs. Firstly, in Figures \ref{fig: bar_visual_control}, \ref{fig: bar_audio_control} and \ref{fig: bar_audiovisual_control}, we can find the results related to the control group: here we can see that the questions related to the visual stimuli differ significantly between random and rhythmic conditions, with higher values for the rhythmic condition. These results are confirmed by the report \textit{p} value in the statistical table in Figure \ref{fig: significance_control_pop}. On the contrary, the bar plot related to the audio stimuli show a small difference between the random and rhythmic condition (always in favor of the rhythmic one) in all the questions, which doesn't result significant. Finally, the results associated with the audiovisual stimuli, depict no significant difference between conditions, sometimes showing the same values for random and rhythmic and in other cases a slightly higher level for the questions related to the random stimuli. 

Moving on to the outcome of the questions answered by the stroke group, which can be found in Figures: \ref{fig: bar_visual_stroke}, \ref{fig: bar_audio_stroke} and \ref{fig: bar_audiovisual_stroke}, quite a few differences can be noticed from the control group. In the bar plot related to the visual inputs, a significant difference between the two conditions can be found in all the questions, even with higher values in the random conditions, compared to the control group. \\
For what concerns the audio stimulus, the conditions result to differ more than in the control group, with a significant \textit{p} value in the first question (Figure \ref{fig: significance_stroke_pop}). Lastly, the questions related to the audiovisual stimulus, received generally higher Likert values compared to the control group, although the difference between random and rhythmic results similar to the one obtained in the other population, with no difference or higher values for the random condition instead of the rhythmic one. 

Later, we decided to correlate the most significant rhythmic question (q2) to the mean power activity of the Occipital, Temporal and Sensorimotor ROIs, dividing them always by population. To represent such correlations we made different scatter plots for each ROI, that can be found from Figure \ref{fig: correlation q2 occipitalROI: control group} for the control group and from Figure \ref{fig: correlation q2 occipitalROI: stroke group} for the stroke group. Additionally, a table with the Spearman coefficient \textit{r} and \textit{p} values can be encountered in Figures \ref{fig: correlation values q2: control} and \ref{fig: correlation values q2: stroke}. \\
Looking at the results for the control group we can observe that the scatter plots result quite similar among ROIs, and that no correlation was found between the question related to the various sensory input and the power brain activity. The data distribution in all the different plots looks spread, with a bigger concentration towards the higher Likert points and the lower values in the brain activity power spectrum. The lack of correlation between the two variables is demonstrated statistically by the Spearman's coefficient and its \textit{p} values. 

Contrarily, looking at the results for the stroke population we can find a significant positive correlation between the values assigned to the behavioral question related to the audio stimuli and the neural activation in the Temporal ROI; while a positive trend correlation in the Sensorimotor ROI. \\
Moving to the question related to the other sensory inputs (visual and audiovisual), no significant correlation with the mean power activity of the different ROIs could be found as in the control group.

Finally, we also aimed to see if a possible correlation between the level of physical activity of the participants and their correspondent mean power activation in their Sensorimotor area could be discovered. \\
The scatter plots of such correlation can be found in Figures \ref{fig: correlation activeq control} and \ref{fig: correlation activeq stroke}, with their correspondent Spearman values in Table \ref{fig: significance correlation activeq}.