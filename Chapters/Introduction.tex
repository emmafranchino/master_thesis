\chapter{Introduction}
%%% general introduction on stroke
%% stroke incidence
An average of 5.5 million people worldwide die every year because of a stroke, ranking it as the second leading cause of death and one of the main cardiovascular diseases \parencite{Donkor_2018}. Research in the United States has stated that more than 795,000 people have a stroke every year, that is almost one person every forty seconds \parencite{Tsao_2023}. In the past three decades, an increase in the incidence of stroke was globally observed, especially in young (18-44 years old) and mid-life (45-64 years old) populations, yet not in older adults (over 65); on the other hand, the incidence of death and disability has decreased as a result of the advance of research and medicine \parencite{Yahya_2020}. \\
Even though a decrease in disability has been observed, stroke remains the leading cause of serious long-term disability: more than 50\% of people who survived a stroke become chronically disabled, both in neurological and physical aspects. One of the most common consequences is long-term motor impairment due to lesions in cortical or sub-cortical motor areas \parencite{Karthikeyan_2019}. 

%% stroke: ischemic and hemorrhagic // cortical subcortical
Stroke is divided in two main subtypes, drawn by the cause of it. The most common one results to be the ischemic stroke, it's produced by a blockage of blood vessels and comprehends TIAs (transient ischemic attacks) or mini-strokes, which only last for a few minutes up to 24 hours. The second type is the hemorrhagic stroke which is caused by bleeding in or around the brain, and it appears to cover just the 15\% of the cases \parencite{Abdu_2021}. \\
On the other hand, if we look at the brain regions lesioned by stroke, it's possible to differentiate it between cortical and subcortical injuries, which can alter different cognitive (e.g. language, space orientation) and physical abilities (e.g. motor impairment, weakness), depending on the specific cerebral area and lobe affected. 

%% Gait impairment
One of the major consequences of experiencing a stroke is impaired gait: a total of 80\% \parencite{Gomez_2020} of patients have an impaired walking ability, with 50\% of patients who are completely unable to walk, 12\% who can walk with assistance, and 37\% can walk independently \parencite{Balaban_2014}. \\
In those cases, gait can result as hemiplegic, slower in speed and cadence and as well as shorter stride length. The balance of the locomotor control is shift due to the lesion in the central nervous system (CNS) affecting automaticity, proprioception, balance, anxiety state and multisensory feedback, which will impair walking abilities (Clark, 2015). \\

%% article Belda-Lois, 2011
Due to the wide spreading of gait impairment in patients who suffered from a stroke, the recovery of walking ability is of the main aims in stroke rehabilitation. Recently rehabilitation techniques have been moving towards top-down approaches, which base the rehabilitation on the state of the brain after stroke. These techniques use a motor learning approach mainly through the integration of sensory feedback, which helps the proprioception of the body thanks to sensory-motor adaption. \\
Auditory feedback has been a fundamental sensory feedback for gait rehabilitation, being used as a real-time cue to correct movement and redirect walking. Together with auditory stimuli, visual ones have been employed in different neurophysiological rehabilitation techniques, such as in Proprioceptive neuromuscular facilitation (PNF) or Motor Imagery (MI). \\
Movement sonification has also been engaged, here the physical properties of the human movement are being translated to audio parameter (e.g. the speed or position of the body are translated into volume or timbre of audio sounds) enhancing motor learning based on motor perception and its representation (Effenberg, 2016).  

Sensory feedback employed for movement rehabilitation induce biological motion perception. Studying the neurological processes that underline motion perception, results as a complex goal, since multiple social cues perception and process are involved. In order to isolate only the interested brain activity related to motor perception a new paradigm studying neural entrainment through electroencephalogram with a frequency-tagging approach, with a neutral visual stimulus of a point-light figure walking, has been proposed by \cite{Cracco_2022}. 

%% neural entrainment: definition
Neural entrainment or sensorimotor synchronization is defined as unidirectional synchronization of neural oscillations to an external rhythmic stimulus \parencite{Lakatos_2019, Haegens_2018}; an action is rhythmically coordinated with a predictable external stimulus \parencite{Pressing_1999}. We use rhythmic synchronization between the auditory and motor systems in many of our everyday behaviors, one of the main being speech. Our brain leans towards a natural synchronization of its movement with endogenous rhythmic signals \parencite{Large_2009}. Such alignment of motor and sensory rhythm at the neural level can be detected in music, dance, sports, rhythmic tasks and verbal communication \parencite{Rosso_2023}. Sensorimotor entrainment can be driven by a status of phase- and frequency-locking of neural oscillations, which will reflect the changes in the external sensory through neuronal excitability  \parencite{Lakatos_2005}. 

%% frequency tagging approach (what is it and why it looks like the best methodology)
A way that has been found successful to study neural entrainment is the frequency-tagging approach, where a brain response is elicited through a repeated cycle movement; if the aim is movement perception related to walking the brain response can be elicited at the end of every footstep \parencite{Cracco_2022}. \\
Brain activity will synchronize with the sensory stimulus, rapidly entraining motor responses: the cognitive and sensory processes present in neural entrainment are reflected in neural oscillation signal obtained through EEG (Thut et al., 2012).

Active sensory and cognitive processes during rhythmic synchronization are represented by neural oscillations that can be observed through electroencephalography (EEG): EEG signals result in time-locked to re-occurring sensory stimuli. \\
State-of-the-art approaches tag frequencies of a rhythmic stimulation in the power spectrum of EEG in order to quantify neural entrainment \parencite{Nozaradan_2011}: the "frequency tagging" methodology converts brain signals into their frequency components. 

%% rhythmic audio and visual stimuli to activate sensorimotor cortex
The usage of frequency tagging with electroencephalography results in the most appropriate technique to study the role of different brain regions integrating multisensory stimuli through neural entrainment, since it appears to be a highly complex process, involving different brain regions. Sometimes, whilst studying neural entrainment with visual or auditory stimuli other processes not primarily related to movement perception are grabbed by neuroimaging technique, due to the stimulus individuals are exposed to \parencite{Oomen_2022}. \\
From previous research, it can be observed that listening to repeated beats activates cortical oscillation evoked from auditory stimuli and anticipations of it \parencite{Snyder_2005}; brain regions that activate during auditory-motor entrainment include: the cerebellum \parencite{Grahn_2011}, inferior colliculus \parencite{Tierney_2013} and basal ganglia \parencite{Thaut_2009}. \\
Nevertheless, the perception of auditory and visual information related to movement is a higher-order cognitive process that necessitates the intricate involvement of multiple brain areas and their interconnections, which include the specific activation of cortical hubs, located at the top cortical networks, that enable sensorimotor integration.

%% aim of the study
In conclusion, research proved that the auditory and visual feedback can elicit motor responses by neural entrainment in the motor cortex, generating synchronized brain oscillations to the rhythm of the stimuli; furthermore different studies demonstrated that such feedbacks play a key role in stroke gait rehabilitation \parencite{Chen_2018}, facilitating motor automaticity, redirection and action.

Even if gait rehabilitation in stroke results much investigated, not many studies have examined the neural activation related to the integration of multisensory feedback. For this reason the aim of the present study is to investigate the brain dynamics supporting the sensorimotor synchronization when coupling auditory and visual inputs related to walking movement, comparing control healthy subjects and patients who suffered from a stroke. \\
The study will be conducted using electroencephalogram frequency tagging approach, aiming to elicit neural entrapment with visual and auditory stimuli at the frequency of 2 Hz, which has been proven to be the most indicated to activate the sensorimotor cortex in healthy subjects by previous study (Matamala-Goméz et al. in preparation).\\
Looking at the results of the previous study conducted with multiple frequencies only on healthy subjects, we expect to find a higher brain activation related to the auditory inputs played at 2 Hz, with an activation of the temporal and sensorimotor cortex. On the other hand, for the visual stimuli we expect an activation of the occipital area and a weaker elicitation of the sensorimotor cortex. For all the sensory inputs we predict to see a significant higher neural synchronization in the stimuli a rhythmic sequence rather than in a random one. \\
Finally, for what concerns the results related to the stroke population, we may see some difference from the control group in the brain activation due to the brain injuries that they've suffered: a lateralized arousal  in the contralateral area of the lesion, or a contrast in the intensity of such arousal could be seen.

% Yet, since cortical and subcortical areas were depicted as facilitators for motor alignment thanks to the ability of beat-based time-keeping \parencite{Cannon_2021}, we expect differences of brain activation in stroke patients compared to healthy subjects, but still, visible neural entrainment elicited by repeated rhythmic multisensory stimuli. 