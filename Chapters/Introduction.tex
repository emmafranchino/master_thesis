\chapter{Introduction}
%%% general introduction on stroke
%% stroke incidence
An average of 5.5 million people worldwide die every year because of a stroke, ranking it as the second leading cause of death and one of the main cardiovascular diseases \parencite{Donkor_2018}. Research in the United States has stated that more than 795,000 people have a stroke every year, that is almost one person every forty seconds \parencite{Tsao_2023}. In the past three decades, an increase in the incidence of stroke was globally observed, especially in young (18-44 years old) and mid-life (45-64 years old) populations, yet not in older adults (over 65); on the other hand, the incidence of death and disability has decreased as a result of the advance of research and medicine \parencite{Yahya_2020}. \\
Even though a decrease in disability has been observed, stroke remains the leading cause of serious long-term disability: more than 50\% of people who survived a stroke become chronically disabled, both in neurological and physical aspects. One of the most common consequences is long-term motor impairment due to lesions in cortical or sub-cortical motor areas \parencite{Karthikeyan_2019}. 

%% stroke: ischemic and hemorrhagic // cortical subcortical
Stroke is divided in two main subtypes, drawn by the cause of it. The most common one results to be the ischemic stroke, it's produced by a blockage of blood vessels and comprehends TIAs (transient ischemic attacks) or mini-strokes, which only last for a few minutes up to 24 hours. The second type is the hemorrhagic stroke which is caused by bleeding in or around the brain, and it appears to cover just the 15\% of the cases \parencite{Abdu_2021}. \\
On the other hand, if we look at the brain regions lesioned by stroke, it's possible to differentiate it between cortical and subcortical injuries. 

% The mechanisms controlling the large-scale temporal coordination of infra-slow activity are unclear, particularly whether specific regions play a leading role in orchestrating global changes in connectivity patterns. A leading hypothesis is that shifts in brain states at rest, or during tasks, depend on highly interconnected cortical regions (hubs), e.g., precuneus, posterior cingulate cortex, and lateral prefrontal cortex, that flexibly interact at different points in time with different networks20–22. However, recent studies have also shown hubs in subcortical regions (basal ganglia23, thalamus23,24, hippocampus20,25–27). Whether cortical synchronization also relies on subcortical regions is still poorly known. Clinical work has shown FC dynamics alterations in a variety of non-focal conditions (neurodegeneration, consciousness abnormalities, schizophrenia, autism)17,19,28–31, which suggests that even more pronounced alterations should occur in focal conditions. Focal lesions, such as those induced by stroke, provide an ideal testbed to study the relations between brain structure and dynamics, since they considerably amplify the natural range of inter-subject variability in anatomical as well as functional connectivity. Subcortical lesions produce widespread functional alterations of the ‘static’ network structure—an anomalous inter-hemispheric segregation and intra-hemispheric integration32–34. Recent studies indicate that lesions induce anomalies also at the dFC level, altering the dynamic balance between integration and segregation35–37. However, which structural changes determine these functional anomalies, and how the interplay between cortex and subcortex contributes to them, has never been thoroughly investigated. (Favaretto, 2022)

%n ---- Jingchun Liu, 2022
% Subcortical ischemic stroke can lead to structural damage to brain regions adjacent to the stroke lesion and to specific regions of cerebral cortex remote from the lesions (Conrad et al., 2021; Liu et al., 2015), possibly by the mechanism of axonal degeneration (Yu et al., 2009). In response to the structural damage caused by subcortical stoke, some cerebral cortical regions can reorganize themselves to facilitate recovery of the impaired neurological function (Brodtmann et al., 2012; Liu, Peng, et al., 2020). Structural damage and reorganization in the cerebral cortex can be assessed quantitatively in vivo using structural magnetic resonance imaging (MRI) followed by calculation of cortical thickness, surface area, and gray matter volume (GMV) from the imaging data (Gupta et al., 2019). In the past decade, many studies have reported structural alterations in the cerebral cortex in chronic patients with subcortical stroke (Cheng et al., 2020; Diao et al., 2017; Jones et al., 2016; Zhang et al., 2014). However, only a limited number of chronic cortical structural changes reported by one study can be replicated in other studies. The small sample sizes and the lack of independent replication may be one reason for this inconsistency across studies. Thus, the chronic structural changes reliably observed in the cortex after subcortical stroke should be investigated in large samples and replicated in independent data sets.

% The corticospinal tract (CST) is the most important white-matter tract for motor output in the human brain, and the impairment of the CST has been associated with motor outcomes in patients with subcortical stroke (Lin et al., 2019; Stinear et al., 2007). Although the CST mainly originates from M1, it also contains contributions from the premotor cortex (PMC), SMA, and primary somatosensory area (S1) (Schieber, 2007; Welniarz et al., 2017). Liu, Wang, et al. (2020) have reconstructed a fine-grained map of the CST fibers showing their different cortical origins, and have associated impairments of different subsets of these CST fibers with different brain structural and functional changes in patients with subcortical stroke. However, the correspondence between early impairment of specific CST fibers and subsequent slow trends in cortical structural parameters has not been systematically investigated in patients with subcortical stroke.

% Many studies have associated cortical structural changes with neurological outcomes (Khlif et al., 2021; Ueda et al., 2019) or motor outcomes (Ueda et al., 2019) in patients with subcortical stroke. Subcortical stroke patients with different degrees of recovery appear to have different patterns of cortical structural change in the chronic stage and different trajectories of evolving cortical change poststroke. For example, stroke patients achieving only partial recovery (PR) show more significant reductions in cortical thickness in the primary motor area than do those with complete recovery (CR; Zhang et al., 2014). However, the differences in the evolution patterns of cortical structural changes between stroke patients with different degrees of recovery remain largely unknown, and such information would be helpful for understanding the mechanisms of neurological recovery after subcortical stroke.

The type of brain injury after suffering a stroke can alter higher and synthetic mental abilities such as language, spatial organization, disturbance in vision or memory loss, depending on the lobe and side of the brain that was impaired. Moreover, a cortical stroke can alter motor and sensory control, often consisting of a hemiparesis affecting the distal extremities more than the proximal, the arm and lower face more than the leg, corresponding to the homunculus, our mental representation of the human body. \\
On the other hand, subcortical strokes (also called small vessel strokes) affect brain structure lying below the cortex, including the thalamus, basal ganglia, brain stem, cerebellum or the internal capsule. The effects of subcortical strokes can often result in weakness and numbness on one side of the body (the opposite one to the stroke) and movement impairment. \\
Since the majority of cases involve cortical strokes, research has mainly focused on this type, not focusing on how injury location can affect the process of recovery from a stroke \parencite{Karthikeyan_2019} at the physical nor neural level, which will try to be assessed later with this study. 

%% incidence of Gait impairment in patients with stroke
As mentioned earlier, motor impairments and upper limb hemiparesis, are much more common in stroke patients, almost 80\% of them are affected; they are generated by damages to cortical regions (e.g. primary motor cortex), subcortical ones (e.g. basal ganglia) or the corticospinal tract. Motor impairments consist of muscle weakness, impaired selective motor control, spasticity, and proprioceptive deficits, which influence gait. The latter appears impaired with a strong asymmetry, which is one of the main aspects of recovery in post-stroke rehabilitation, with an efficiency of almost 50\% of patients reacquiring their ability to walk independently. Since post-stroke gait dysfunction is one of the most investigated neurological gait disorders, various rehabilitative methods have been used, but the specific roles of the brain in these processes still need to be completely assessed \parencite{Balaban_2014}. 

%% Marta comment: Before introducing neural entrainment, you should add information regarding the use of audiovisual stimuli to integrate walking movement as rehabilitation technique. Then, you introduce neural entrainment as a the neural mechanism behind the integration or cortical processing of sensory stimuli

%% neural entrainment: definition
Neural entrainment or sensorimotor synchronization is defined as unidirectional synchronization of neural oscillations to an external rhythmic stimulus \parencite{Lakatos_2019, Haegens_2018}; an action is rhythmically coordinated with a predictable external stimulus \parencite{Pressing_1999}. We use rhythmic synchronization between the auditory and motor systems in many of our everyday behaviors, one of the main being speech. Our brain leans towards a natural synchronization of its movement with endogenous rhythmic signals \parencite{Large_2009}. Such alignment of motor and sensory rhythm at the neural level can be detected in music, dance, sports, rhythmic tasks and verbal communication \parencite{Rosso_2023}. Sensorimotor entrainment can be driven by a status of phase- and frequency-locking of neural oscillations, which will reflect the changes in the external sensory through neuronal excitability  \parencite{Lakatos_2005}. 

%% frequency tagging approach (what is it and why it looks like the best methodology)
% HERE CITE CRACCO \Parencite{Cracco_2022}
Research investigating neural entrainment has usually employed the finger-tapping paradigm with visual and auditory stimuli repeated isochronously, which neurotypical individuals would entrain their movement to; motor entrainment seems to occur due to the prediction of future events, based on the first two or three repetitions \parencite{Schmidt_1997}. \\
Active sensory and cognitive processes during rhythmic synchronization are represented by neural oscillations that can be observed through electroencephalography (EEG): EEG signals result in time-locked to re-occurring sensory stimuli. \\
State-of-the-art approaches tag frequencies of a rhythmic stimulation in the power spectrum of EEG in order to quantify neural entrainment \parencite{Nozaradan_2011}: the "frequency tagging" methodology converts brain signals into their frequency components. 

%% rhythmic audio and visual stimuli to activate sensorimotor cortex
The usage of frequency tagging with electroencephalography results in the most appropriate technique to study the role of different brain regions integrating multisensory stimuli through neural entrainment, since it appears to be a highly complex process, involving different brain regions. Sometimes, whilst studying neural entrainment with visual or auditory stimuli other processes not primarily related to movement perception are grabbed by neuroimaging technique, due to the stimulus individuals are exposed to \parencite{Oomen_2022}. \\
From previous research, it can be observed that listening to repeated beats activates cortical oscillation evoked from auditory stimuli and anticipations of it \parencite{Snyder_2005}; brain regions that activate during auditory-motor entrainment include: the cerebellum \parencite{Grahn_2011}, inferior colliculus \parencite{Tierney_2013} and basal ganglia \parencite{Thaut_2009}. \\
Nevertheless, the perception of auditory and visual information related to movement is a higher-order cognitive process that necessitates the intricate involvement of multiple brain areas and their interconnections, which include the specific activation of cortical hubs, located at the top cortical networks, that enable sensorimotor integration.

Recognizing the ability of the brain to synchronize neural activity with the rhythms of external stimuli, our study intended to elicit such entrainment by auditory and visual stimuli in individuals who suffered from a stroke, so that later these stimuli would facilitate the process of gait rehabilitation, activating spontaneously the motor cortex and enhancing movement correspondent to their rhythm. Furthermore, research has demonstrated that sensory inputs are crucial in motor rehabilitation, underlying how impactful they are for post-stroke recovery \parencite{Chen_2018}.\\
Yet, since cortical and subcortical areas were depicted as facilitators for motor alignment thanks to the ability of beat-based time-keeping \parencite{Cannon_2021}, we expect differences of brain activation in stroke patients compared to healthy subjects, but still, visible neural entrainment elicited by repeated rhythmic multisensory stimuli. 

%% 2 Hz part: the stimuli were previously used in a pilot study, only with healthy participants that aimed to discover at which frequency the stimuli presented were able to activate cerebral regions of interest that could integrate them. For such goal, in the previous research all the stimuli were played at different frequencies (1, 2, 3.6 Hz), whilst in the present one, they were only used at the frequency of 2 Hz, which appeared to be the best fitting one for neural entrainment. 

%%% --- aim of the study --- %%%
The present study aims to precisely determine the role of the brain affected by a cortical or subcortical stroke in the integration of different stimuli that trigger neural entrainment. 
% The introduction should be improved. We have to explain that we are showing the sensory inputs at 2 Hz. based on previous research. Because 2 Hz frequency seems the most indicated to activate the sensorimotor cortex in healthy subjects. Then we have to explain that the aim of this study is to compare the cortical integration or brain dynamics when presenting audio and visual stimuli related to walking movement at to 2 Hz frequency.

% Marta article: The aim of this study is to investigate brain dynamics supporting rhythmic sensorimotor synchronization when coupling auditory and visual inputs related to walking movement presented at different frequencies 1 2 and, 3 6 Hz) using an EEG frequency tagging approach.

%%% --- useful for the results --- %%%
% Neural oscillation signals obtained through electroencephalography (EEG) consistently reflect specific cognitive and sensory processes (Thut et al., 2012). Frequencies in the delta (1–3 Hz), theta (4–7 Hz) and alpha (8–12 Hz) bands are shown to be part of a hierarchical network of oscillators involved in predictive sensory processing (Arnal et al., 2014). Delta oscillations are associated with decision making and attentional processes (Güntekin and Başar, 2016), while theta band activity is associated with cognitive control across brain networks Cavanagh and Frank, 2014). Frequencies in the alpha range are associated with entrainment using visual stimuli (Keitel et al., 2014; Spaak et al., 2014). Beta activity (13–30 Hz) is associated with anticipation and predictive timing (Arnal et al., 2011), and gamma band activity (30–80 Hz) is associated with top-down attentional control (Debener et al., 2003; Tiitinen et al., 1993)

% Additionally, across segments, the degree of phase-locking for frequency-specific EEG oscillations can be measured (Lachaux et al.,  1999; Roach and Mathalon, 2008). When data are summed across trials, the activity represents evoked and induced responses. To obtain a measure of induced activity, the evoked activity is subtracted from the summed activity. 

% At the basic level, neurons in the primary auditory cortex have a dual representation of the repetition rate of a stimulus, by discharge rate, and by synchronized firing patterns of a group of neurons (Bendor and Wang, 2007; Eggermont, 2001).
% The observation that target frequencies dominate the spectrum has been commonly taken as evidence for the underlying entrainment of neural oscillations (e.g., Nozaradan et al., 2011 ; Nozaradan et al., 2012 ; Lenc et al., 2018 ), which is not exempt from critiques