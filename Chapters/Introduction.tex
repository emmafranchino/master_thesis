\chapter{Introduction}
%%% general introduction on stroke
%% stroke incidence
An average of 5.5 million people worldwide die every year because of a stroke, ranking it as the second leading cause of death and one of the main cause of cardiovascular diseases \parencite{Donkor_2018}. Research in the United States has stated that more than 795,000 people have a stroke every year, that is almost one person every forty seconds \parencite{Tsao_2023}. In the past three decades, an increase in the incidence of stroke was globally observed, especially in young (18-44 years old) and mid-life (45-64 years old) populations, yet not in older adults (over 65); on the other hand, the incidence of death and disability has decreased as a result of the advance of research and medicine \parencite{Yahya_2020}. \\
Even though a decrease in disability has been observed, stroke remains the leading cause of serious long-term disability: more than 50\% of people who survived a stroke become chronically disabled, both in neurological and physical aspects. One of the most common consequences is long-term motor impairment due to lesions in cortical or subcortical motor areas \parencite{Karthikeyan_2019}. 

%% stroke: ischemic and hemorrhagic // cortical subcortical
Stroke is divided in two main subtypes, drawn by the cause of it. The most common one results to be the ischemic stroke, it's produced by a blockage of blood vessels and comprehends TIAs (transient ischemic attacks) or mini strokes, which only last for a few minutes up to 24 hours. The second type is the hemorrhagic stroke which is caused by bleeding in or around the brain, and it appears to cover just the 15\% of the cases \parencite{Abdu_2021}. \\
On the other hand, if we look at the brain regions lesioned by stroke, it's possible to differentiate it between cortical and subcortical injuries, which can alter different cognitive (e.g. language, space orientation) and physical abilities (e.g. motor impairment, weakness), depending on the specific cerebral area and lobe affected. 

%% Gait impairment
One of the major consequences of experiencing a stroke is impaired gait: a total of 80\% of patients have an impaired walking ability, with 50\% of patients who are completely unable to walk, 12\% who can walk with assistance, and 37\% can walk independently \parencite{Balaban_2014}. In those cases, gait can result as hemiplegic, slower in speed and cadence and as well as shorter stride length \parencite{Gomez_2020}. The balance of the locomotor control is shift due to the lesion in the central nervous system (CNS) affecting automaticity, proprioception, balance, anxiety state and multisensory feedback, which will impair walking abilities \parencite{Clark_2015}. 

Due to the wide spreading of gait impairment in patients who suffered from a stroke, the recovery of walking ability is of the main aims in stroke rehabilitation. Recently rehabilitation techniques have been moving towards top-down approaches, which base the rehabilitation on the state of the brain after stroke. These techniques use a motor learning approach mainly through the integration of sensory feedback, helping the proprioception of the body thanks to sensory-motor adaption \parencite{Belda-Lois_2011}. \\
The human brain presents the function of multisensory integration: it integrates the information of different sensory modalities into a coherent representation \parencite{Stein_2008}. Multisensory integration can improve motor responses, through the ability of the brain to extract sensory information regarding body position and anticipate future positions. Additionally, it has been observed that many multisensory processes appear to be largely preserved in stroke patients, keeping integrate the ability to synthesize multisensory information representing motor activity \parencite{Bolognini_2013}.

From on side, research has investigated the usage of different sensory inputs for post-stroke gait rehabilitation: for what concerns visual stimuli, it has been shown that Mirror Neuron System (MNS) play a crucial role in action observation and motor learning, through the stimulation of an internal model, able to reinforce sensorimotor representation \parencite{Rizzolati_2004}. Furthermore, visual feedbacks used in motor imagery-based training can also improve motor recovery in post-stroke patients, especially the ones affected by severe hemiparesis \parencite{Mihara_2013}. \\
On the other hand, auditory feedback has been a fundamental sensory feedback for gait rehabilitation, being used as a real-time cue to correct movement and redirect walking ability. Sounds can activate excitability through auditory-motor circuits at the reticulo-spinal level; if they are repeated rhythmically, they can entrain the timing of muscle activation, enhancing movement during rhythmic actions \parencite{Thaut_1999}. \\
Lastly, the employment of audiovisual stimuli has been explored: MNS appears to respond to the visual representation of action, as well as the auditory one \parencite{Rizzolati_2004}; the integration of both sensory modalities can result more beneficial considering the positive effects of multisensory integration, both in healthy and clinical populations \parencite{Bolognini_2015}. \\
Both auditory and visual stimuli have been employed in different neurophysiological rehabilitation techniques, such as in Proprioceptive Neuromuscular Facilitation (PNF) \parencite{Moros_2000} or Motor Imagery (MI) \parencite{Mason_2007}. 

Some evidence has found auditory feedback as fundamental for walking rehabilitation: it reacts faster than the visual system and can continuously be delivered without constraining the movements. From such discoveries, movement sonification has been engaged for these goals; here the physical properties of the human movement are being translated to audio parameter (e.g. the speed or position of the body are translated into volume or timbre of audio sounds) \parencite{Effenberg_2016}, enhancing motor learning based on motor perception and its representation \parencite{Bevilacqua_2016}. \\
Still, little research on movement sonification for sensorimotor learning and movement recovery has been done, even more the study of the neurological processes underlying such phenomena in clinical population.

Sensory feedback employed for movement rehabilitation induce biological motion perception. Studying the neurological processes that underline motion perception, results as a complex goal, since multiple social cues perception and process are involved. In order to isolate only the interested brain activity related to motor perception a new paradigm studying neural entrainment through electroencephalogram with a frequency-tagging approach, with a neutral visual stimulus of a point-light figure walking, has been proposed by \cite{Cracco_2022}. 

%% neural entrainment: definition
Neural entrainment or sensorimotor synchronization is defined as unidirectional synchronization of neural oscillations to an external rhythmic stimulus \parencite{Lakatos_2019, Haegens_2018}; an action is rhythmically coordinated with a predictable external stimulus \parencite{Pressing_1999}. We use rhythmic synchronization between the auditory and motor systems in many of our everyday behaviors, one of the main being speech. Our brain leans towards a natural synchronization of its movement with endogenous rhythmic signals: auditory rhythms rapidly entrain motor responses into stable steady synchronization states \parencite{Thaut_2003}. Such alignment of motor and sensory rhythm at the neural level can be detected in music, dance, sports, rhythmic tasks and verbal communication \parencite{Rosso_2023}. Sensorimotor entrainment can be driven by a status of phase- and frequency-locking of neural oscillations, which will reflect the changes in the external sensory through neuronal excitability \parencite{Lakatos_2005}. 

%% frequency tagging approach (what is it and why it looks like the best methodology)
A way that has been found successful to study neural entrainment is the frequency-tagging approach, where a brain response is elicited through a repeated cycle movement; if the aim is movement perception related to walking the brain response can be elicited at the end of every footstep \parencite{Cracco_2022}. The cognitive and sensory processes present in neural entrainment (brain synchronization with the sensory input presented) are represented by neural oscillations that can be observed through electroencephalography (EEG): EEG signals result in time-locked to re-occurring sensory stimuli \parencite{Thut_2012}. \\
State-of-the-art approaches tag frequencies of a rhythmic stimulation in the power spectrum of EEG in order to quantify neural entrainment \parencite{Nozaradan_2011}: the "frequency tagging" methodology converts brain signals into their frequency components. 

%% rhythmic audio and visual stimuli to activate sensorimotor cortex
The usage of frequency tagging with electroencephalography results in the most appropriate technique to study the role of different brain regions integrating multisensory stimuli through neural entrainment, being able to isolate the brain processes related to movement perception, distinguishing them from other cognitive processes that may be related to social cues using neutral sensory inputs, such as a point-light figure with no gender, weight nor emotional trait \parencite{Cracco_2022}. \\
Brain synchronization remains a complex phenomenon, previous research showed that listening to repeated rhythmic beats activates cortical oscillation evoked from auditory stimuli and anticipations of it \parencite{Snyder_2005}. The different brain regions that are being enhanced during auditory-motor entrainment include: the cerebellum \parencite{Grahn_2011}, inferior colliculus \parencite{Tierney_2013}, basal ganglia \parencite{Thaut_2009} and cortical hubs, located at the top cortical networks, that enable sensorimotor integration. \\
%% aim of the study
In conclusion, research proved that the auditory and visual feedback can elicit motor responses by neural entrainment in the motor cortex, generating synchronized brain oscillations to the rhythm of the stimuli; furthermore different studies demonstrated that such feedbacks play a key role in stroke gait rehabilitation \parencite{Chen_2018,Bolognini_2016}, facilitating motor automaticity, redirection and action. 

Even if more research in gait rehabilitation has been done in patients with stroke, few investigations have examined the neural activation related to the integration of multisensory stimuli (audio and visual) related to walking movement. Hence, the aim of the present study is to investigate the brain dynamics supporting the sensorimotor synchronization when coupling auditory (footstep sounds) and visual (walking point-light-figure) inputs (see Figure \ref{fig: visual stimuli}) related to walking movement in a group of patients with stroke compared to a group of healthy subjects. \\
The study will be conducted using an electroencephalogram frequency tagging approach, aiming to elicit neural synchronization between brain oscillatory activity and the presented sensory inputs (visual: walking point-light-figure; and auditory: footstep sounds) related to walking ability presented at 2 Hz. The frequency rate presentation of the stimuli was selected based on previous study conducted in our laboratory by Matamala-Gomez et al. in which the authors observed that at 2 Hz when presenting auditory and visual inputs at rhythmic sequences there was a higher synchronization with the sensorimotor cortex in healthy subjects (Matamala-Gomez et al. in preparation). In detail, in the previous study the authors compared different frequency rates on the presentation of the sensory inputs (1, 2, and 3.6 Hz) in rhythmic and random sequences, and observed a higher synchronization with the temporal and sensorimotor cortex when presenting auditory (footstep sounds) and visual (walking point-light-figure) at 2 Hz in rhythmic sequences. Moreover, the authors also observed a higher synchronization with the Occipital cortex when presenting the visual input (walking point-light-figure) at 2 Hz in rhythmic sequences. Based on the previous results, in the present study we expect to find a higher brain synchronization when presenting auditory inputs (footstep sounds) at 2 Hz in rhythmic sequences, with the temporal and sensorimotor cortex. Moreover, we also expect to find a higher synchronization when presenting the visual inputs (walking-point-light figure) at 2 Hz in rhythmic sequences with the occipital cortex and a weaker elicitation of the sensorimotor cortex. For all the sensory inputs we predict to see a significant higher neural synchronization in the stimuli a rhythmic sequence compared to the random sequences. \\
Finally, regarding the patients with stroke, we expect to observe some differences in brain synchronization compared to the healthy control group. Since cortical and subcortical areas were depicted as facilitators for motor alignment thanks to the ability of beat-based time-keeping \parencite{Cannon_2021}, we expect differences of brain activation in stroke patients compared to healthy subjects, but still, visible neural entrainment elicited by repeated rhythmic multisensory stimuli.

%%% --- poster Marta
% Frequency tagging measures a periodic change in voltage amplitude in the electrical activity recorded on the human scalp by using EEG when presenting stimuli repeated at a fixed rate1. Regarding auditory stimulation, some studies show a tonic synchronization response when using periodic auditory stimulation set at 2Hz2. Further, others found brain responses associated with biological motion perception by showing a point-light walker video moving at a pace of 2.4Hz3.
% Movement sonification implies the mapping of movement signals into sound, that can be used for motor rehabilitation4. To date, none have investigated frequency tagging responses and the neural correlates related to movement sonification techniques for motor rehabilitation.
% Aim of the study
% In this study we aimed to evaluate the impact of sonification on walking ability as indexed by neural markers and brain dynamics on the sensorimotor, the auditory, and the visual cortex using a frequency tagging approach in healthy subjects.